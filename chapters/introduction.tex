\chapter{Introduction}

\todo{Configure minted to not use green color for Vec, Box, etc to make sure that all types look consistent}

Rust is a modern, open-source programming language with a focus on memory safety and performance. Its rich type system not only eliminates several classes of bugs but also makes the language powerful and expressive for building high-level programs such as web servers and command-line interface applications. With direct access to computer memory and hardware, Rust is an excellent language for embedded and low-level programming as well.

However, due to the emphasis on memory safety and rigorous checks enforced by the compiler, it is common to get errors when compiling a Rust program. One of the rules is the forbidden simultaneous sharing and mutation of an object. Often, errors caused by this rule can be avoided by better design, but sometimes, the best resolution is to clone the value before sharing it. Naive cloning by copying, however, is an expensive operation both in terms of time and space. Thus, resorting to it as a solution, especially for large-sized collections might be in-efficient.

Persistent data structures are the data structures that provide access to all their previous versions. Often persistence is achieved through copying, and thus, various data structure designs have been developed throughout the years to optimize for this operation. The standard library of the Scala programming language\footnote{\url{https://www.scala-lang.org/}} provides a persistent vector implementation that is efficient not only when copied but also across all operations.

This thesis presents \pvecrs{}, a project that contributes a vector implementation with efficient clone operation that borrows ideas from persistent data structures. The project explores novel approaches to optimize the vector’s performance by leveraging the type system of Rust, as well as aiming to achieve convenient, idiomatic interface familiar to developers. The proposed optimizations are evaluated and discussed based on the results of the sequential and parallel tests.

In this introductory chapter, we will look at the background for the \pvecrs{} project. First, in section \ref{sec:rust}, we will look at the Rust programming language. Then, in section \ref{sec:psds}, we will take a closer look at persistent data structures and their categories. Finally, section \ref{sec:contributions} gives an overview of the contributions made in this project.

\section{The Rust programming language}
\label{sec:rust}

Rust is a relatively new programming language developed at Mozilla that favors reliability and performance. It is a statically typed language with type inference, with high-level constructs such as closures, pattern matching, and algebraic data types. At the same time, Rust gives the option to control low-level details, such as memory management, without all the unsafety traditionally associated with it.

Rust does not have a garbage collector and can be configured without the standard library, allowing it to be used for programming microcontrollers, operating systems, and drivers, the domain that has been occupied mainly by C/C++. It avoids the tradeoff between control and safety by statically checking memory correctness at compile-time without introducing any runtime overhead \cite{reed-patina}.

The advanced type system of Rust guarantees memory safety by enforcing ownership and borrowing rules, making Rust a unique programming language that helps to write fast, safe and reliable software.

The project presented in the thesis, \pvecrs{}, contributes the persistent vector to the Rust programming language. The following sections are a basic introduction to the relevant parts of Rust touched in this thesis. For a more in-depth overview of the more advanced language features, see the Rust book\footnote{\url{https://doc.rust-lang.org/book/}}.

\subsection{Ownership and borrowing}

\paragraph{Ownership}
Ownership is Rust's most unique feature, and it enables Rust to make memory safety guarantees without needing a garbage collector. Every object allocated in Rust always has an \emph{owner}. Ownership can be transferred from one function to another by \emph{moving} the object. Rust's compiler is capable of tracking these movements and identify the location where the object is no longer used, or in Rust's terminology, where it goes out of scope, and generate code for destroying that object. To summarize:

\begin{itemize}
    \item Each value in Rust has a variable that is called its owner.
    \item There can only be one owner at a time.
    \item When the owner goes out of scope, the value will be dropped.
\end{itemize}

\paragraph{Borrowing}
Borrowing in Rust is the act of creating references to an object. The compiler checks that references always go out of scope before the object they are pointing to, returning an error if it does not, thus guaranteeing that references will never be dangling pointers \cite{reed-patina}. The borrowing rules are:

\begin{itemize}
    \item At any given time, you can have either one mutable reference or any number of immutable references.
    \item References must always be valid: they cannot outlive an object they are pointing to, and they have to point to an object of a correct type.
\end{itemize}

Ownership and borrowing rules are demonstrated in listing \ref{lst:ownership-and-borrowing} that was taken from \cite{linus-parallelization}. On line 2, a new vector is created and assigned to the \mintinline{rust}{vec} variable. Since \mintinline{rust}{vec} owns the object, it is allowed to mutate it by pushing a value on line 3.

\begin{listing}[!ht]

    \centering
    \begin{minted}[
        fontsize=\small,
        stripnl=false,
        framesep=4mm,
        frame=lines,
        autogobble,
        linenos
    ]{rust}
        fn main() {
            let mut vec = Vec::new();
            vec.push(42);

            { // Two references are created.
                let vecref = &vec;
                borrow(&vec);
                // ^ Ref goes out of scope when borrow returns
            } // vecref goes out of scope here

            borrow_mut(&mut vec); // <- Lend ‘vec‘ mutably to ‘borrow_mut‘
            take(vec); // <- Ownership transferred to ‘take‘
            vec.push(37); // <- Error: use of moved value: ‘vec‘
        }

        fn take(mut my_data: Vec<i32>) {
            my_data.push(99); // ‘my_data‘ is owner, can perform mutation
        } // ‘my_data‘ goes out of scope and will be freed here

        fn borrow(vec: &Vec<i32>) {
            vec.push(10); // error: cannot borrow immutable borrowed ...
            let element = vec.get(0); // Read is possible.
        } // Borrowing ends, but ‘vec‘ continues to live in ‘foo‘

        fn borrow_mut(vec: &mut Vec<i32>) {
            vec.push(0); // Mutable borrow of ‘vec‘, can mutate.
        }
    \end{minted}

    \caption{Demonstrating ownership and borrowing rules.}
    \label{lst:ownership-and-borrowing}
\end{listing}

On lines 6 and 7, two references are created with the \mintinline{rust}{&} operator. One of them is borrowed by the \mintinline{rust}{borrow} function, with the \mintinline{rust}{vec} still being the owner. When \mintinline{rust}{borrow} returns, the reference goes out of the scope first, and then the same happens to \mintinline{rust}{vecref} a line later. The two references are a demonstration of the possibility to simultaneously create more than one immutable reference to the same object.

On line 10 a new mutable reference is created with the \mintinline{rust}{&mut} operator. While data is borrowed mutably, no other references can exist. When \mintinline{rust}{borrow_mut} returns the reference goes of the scope.

When the function \mintinline{rust}{take} is invoked the ownership to the \mintinline{rust}{vec} object is transferred to the variable \mintinline{rust}{my_data}. The \emph{move} happens because \mintinline{rust}{take} accepts the object by value rather than by a reference. Compiler will error on attempt to use \mintinline{rust}{vec} after it has been moved on line 13.

\paragraph{Unsafe Rust}
Rust has a second language hidden inside it that does not enforce memory safety guarantees: it is called \emph{unsafe} Rust. Rust forces developers to find a safer solution to problematic code or mark it as unsafe and thus highlight the code as a potential source of problems during future debugging.

The \pvecrs{} project was designed in the way that it does not require unsafe Rust features. It was done intentionally to minimize the risk of introducing bugs.

\subsection{Memory management: the stack and the heap}
In programming languages such as Java, there is no distinction between the stack and the heap memory, because language abstracts them away from a developer. However, in a systems programming language like Rust, whether a value is on the stack or the heap affects the language behavior.

The stack is very fast and is where memory is allocated in Rust by default. The allocation is local to a function call and is limited in size. The heap, on the other hand, is slower and is explicitly allocated by program, but it is virtually unlimited in size and is globally accessible \cite{rust-book-2e}.

The following section gives an overview of how Rust supports heap allocation.

\subsubsection*{Smart pointers}
\emph{Smart pointers} are data structures that not only act as a pointer but also have additional metadata and capabilities.

\begin{listing}[!ht]

    \centering
    \begin{minted}[
        fontsize=\small,
        stripnl=false,
        framesep=4mm,
        frame=lines,
        autogobble,
        linenos
    ]{rust}
        fn main() {
            let b = Box::new(5); // <- 5 will be stored on the heap.
            println!("b = {}", b);
        } // <- Goes out of the scope.
    \end{minted}

    \caption{Example of using box smart pointer.}
    \label{lst:box}
\end{listing}

\paragraph{Box}
The most straightforward smart pointer is \type{Box<T>} that allows storing data on the heap rather than the stack.

When box pointer in listing \ref{lst:box} goes out of the scope on line 4, corresponding heap allocation gets freed as well.

\paragraph{Rc or reference counting}
There are cases when a single object might have multiple owners. In graph data structures, multiple edges might point to the same node, and that node is conceptually owned by all of the edges that point to it. A node should not be cleaned up unless it does not have any edges pointing to it \cite{rust-book-2e}.

\begin{listing}[!ht]

    \centering
    \begin{minted}[
        fontsize=\small,
        stripnl=false,
        framesep=4mm,
        frame=lines,
        autogobble,
        linenos
    ]{rust}
        fn main() {
            let mut rc = Rc::new(Vec::new());
            // ^ Wrapping the vec instance into rc pointer

            Rc::make_mut(&mut rc).push(42);
            // ^ Ref count is 1, push succeeds without cloning.

            take(rc.clone()); // <- Ref count is incremented to 2.
            Rc::make_mut(&mut rc).push(37); // <- Updating the original 'vec'.
        } // <- 'rc' is decremented to 0 and dropped

        fn take(mut my_rc: Rc<Vec<i32>>) {
            println!("val={:?}", my_rc.get(0));
            // ^ Reading values does not cause clone

            Rc::make_mut(&mut my_rc).push(99);
            // ^ Mutating the object with ref count of 2
            // will clone the wrapped value and then update it.
        } // ‘my_rc' is decremented to 0 and dropped.
    \end{minted}

    \caption{Example of using reference counting pointer.}
    \label{lst:rc}
\end{listing}

To enable multiple ownership, Rust has a type called \type{Rc<T>}, which is an abbreviation for reference counting. The \type{Rc<T>} type keeps track of the number of references to a value that determines whether or not a value is still in use. If there are zero references to a value, the value can be cleaned up without any references becoming invalid.

Listing \ref{lst:rc} demonstrates how multiple ownership works using the \type{Rc<T>} pointer. On line 2, a vector is created and wrapped into reference counting pointer immediately. On the first call to \mintinline{rust}{Rc::make_mut} method, the number 42 is pushed onto the vector.

The \mintinline{rust}{rc} variable is cloned before it is passed to \mintinline{rust}{take} by value, causing the reference count to be incremented to 2. It is important to note that the underlying vector object is not cloned until line 16, where \mintinline{rust}{Rc::make_mut} function clones the inner value to ensure unique ownership. When \mintinline{rust}{take} is finished, both \mintinline{rust}{my_rc} and the newly copied vector go out of the scope and destroyed.

The execution continues on line 9, where \mintinline{rust}{main} proceeds updating the vector. In this case, \mintinline{Rust}{Rc::make_mut} is not causing a clone because the underlying vector has only a single reference to it.

Note that \type{Rc<T>} is only for use in single-threaded scenarios. \type{Arc<T>} that stands for atomic reference counting should be used in a multi-threaded environment instead.

\subsection{Enums, structs and traits}

\subsubsection*{Structs}

\begin{listing}[!htbp]

    \centering
    \begin{minted}[
        fontsize=\small,
        stripnl=false,
        framesep=4mm,
        frame=lines,
        autogobble,
        linenos
    ]{rust}
        struct Book {
            author: String,
            name: String
        }
    \end{minted}

    \caption{A basic Rust struct}
    \label{lst:struct}
\end{listing}

Structs are the most common way to declare complex custom data types in Rust without introducing runtime overhead. A \emph{struct}, or \emph{structure}, is a custom data type that lets you name and package together multiple related values that make up a meaningful group\footnote{\url{https://doc.rust-lang.org/book/ch05-00-structs.html}}.

\subsubsection*{Enums}

\begin{listing}[!htbp]

    \centering
    \begin{minted}[
        fontsize=\small,
        stripnl=false,
        framesep=4mm,
        frame=lines,
        autogobble,
        linenos
    ]{rust}
        struct Point(isize, isize);

        enum Shape {
            Rectangle {
                p_1: Point,
                p_2: Point
            },
            Triangle {
                p_1: Point,
                p_2: Point,
                p_3: Point
            }
        };
    \end{minted}

    \caption{A basic Rust enum}
    \label{lst:enum}
\end{listing}

\emph{Enums} or \emph{enumerations}, are used to define a custom data type by enumerating its possible variants, where each variant might have different data associated with it. The size of an enum instance is equal to the maximum size of its variants. For example, \mintinline{rust}{Shape::Rectangle} will be allocated the same amount of memory as \mintinline{rust}{Shape::Triangle}, even though it contains only two points.

\subsubsection*{Traits}

\begin{listing}[!htbp]

    \centering
    \begin{minted}[
        fontsize=\small,
        stripnl=false,
        framesep=4mm,
        frame=lines,
        autogobble,
        linenos
    ]{rust}
        struct Email(String);
        struct Tweet(String);

        trait Message {
            fn send();
        }

        impl Message for Email {
            fn send() { /* sending email */ }
        }

        impl Message for Tweet {
            fn send() { /* sending tweet */ }
        }
    \end{minted}

    \caption{A basic Rust trait}
    \label{lst:trait}
\end{listing}

A \emph{trait} is a language feature that tells the Rust compiler about functionality a type must provide. Conceptually it is similar to an interface in Java. One of the trait's superpowers is the ability to implement them for foreign types. Traits support default function implementations, as well as inheritance. Listing \ref{lst:trait} shows how traits can be used to define the common contract between several data types.

\section{Persistent data structures}
\label{sec:psds}

\emph{Ephemeral} data structures are data structures that do not keep the history of their versions. Once an ephemeral data structure is modified, there is no mechanism to go back to previous states. This behavior is typical for collections provided by the standard library of modern general-purpose programming languages.

\emph{Persistent} data structures, on the other hand, are set up in a different way to allow access to any version, old or new, at any time \cite{making-data-structures-persistent}. Persistent data structures were adopted in the functional programming languages such as Scala\footnote{\url{https://www.scala-lang.org/}} and Clojure\footnote{\url{https://clojure.org/}}.

\subsection{Persistence categories}
Persistent data structures are categorized based on the operations which they offer over their versions:
\begin{itemize}
    \item \textit{Partial persistence} --- Read-only access to any previous version of the data structure with the ability to update only the newest one. The versions are ordered linearly.
    \item \textit{Full persistence} --- Read and write operations are available for all versions. The versions are structured as a tree.
    \item \textit{Confluent persistence} --- Full persistence with ability to merge several previous versions into a single new one. The versions form a directed acyclic graph \cite{fully-persistent-lists-with-catenation}.
    \item \textit{Functional persistence} --- This model takes its name from functional programming, where objects are immutable. In comparison to the previous models, it prohibits change of the internal representation of the data structure \cite{purely-functional-data-structures}.
\end{itemize}

\subsection{Achieving persistence}
While persistence can be achieved by copying, the cost of write operations quickly becomes unacceptable. Functional programming languages that have immutable data structures by design were at the forefront of the research for a more efficient alternative.

Multiple data structure designs were proposed, often offering good performance for particular operations with the focus on particular use cases. For example, a singly linked list guarantees \bigo{1} performance for adding or removing elements at the head, but suffers from the worst case \bigo{n} random access operations.

\subsection{Tries}
Since most write operations modify only some parts of a data structure, a complete copy is often unnecessary. A better approach is to use the similarity between the new and old versions by \emph{sharing} structure between them. For example, instead of using a single memory block to represent a data structure, it can be split into smaller pieces or \emph{nodes} linked together as a \emph{tree}. Since modifications only apply to some nodes, the rest of them remain unchanged and can be shared without copying.

Inspired by Bagwell's paper Ideal Hash Trees\cite{ideal-hash-trees}, Rick Hickey pioneered the first persistent vector to offer uniformly good performance across different operations comparable to mutable vectors for the Clojure programming language. The Clojure's vector is \emph{fully} persistent, as it allows us to read and write to all versions.

The persistent vector is a \emph{bit-array mapped trie}, or simply a "wide" tree with a high branching factor. Thus, tries are very shallow, being at most 7 levels deep when the branching factor is equal to 32, offering effectively constant time for almost all operations. Wide nodes and shallow height also help to improve cache locality and to reduce the cache misses.

Later, Bagwell and Rompf introduced \emph{relaxed radix balanced tree} that was based on the design of the Clojure's persistent vector \cite{efficient-immutable-vectors}. \treerrb{} offers efficient concatenation and splitting, enabling vectors to be suitable data structures for parallel processing. Later it became a foundation for the parallel vector in the Scala's standard library \cite{rrb-vector-practical-general-purpose-im-sequence}.

The \treerrb{} data structure is used to implement the persistent vector in \pvecrs{} and is explored further in chapter \ref{chapter:persistent-vector}.

\section{Goals and contributions}
\label{sec:contributions}
The purpose of this thesis is to explore and evaluate novel ideas for optimizations that emerge at the intersection of the unique Rust features and persistent data structures to contribute a persistent vector implementation with excellent performance and idiomatic Rust interface.

The following contributions are made in this thesis:
\begin{itemize}
    \item The persistent vector implementation based on \treerrb{} with efficient concatenation and splitting to make it a reasonable data structure to use for parallel programming.
    \item An extension for the data parallelism library -- Rayon, fused with the persistent vector to create an efficient general-purpose parallel vector.
    \item Describing and implementing novel optimizations specific to Rust, specifically: \emph{dynamic representation} and \emph{unique access} optimizations.
    \item The extensive performance evaluation of the contributed persistent and parallel vectors, as well as the effectiveness of the proposed optimizations.
\end{itemize}

Chapter \ref{chapter:radix-balanced-tree} gives an overview of the \treerrb{} data structure that serves as the foundation for the persistent vector. In chapter \ref{chapter:persistent-vector}, the \pvecrs{} project, its interface, and optimizations specific to Rust are presented in detail. The chapter \ref{chapter:methodology} focuses on the performance evaluation of the persistent vector followed by the chapter with results.
