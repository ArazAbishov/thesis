\chapter{Performance evaluation}
In this chapter, I will introduce a methodology for performance evaluation of \rrbvec{}, \pvec{}, and their variants, in comparison to implementations from \imrsvec{} and Rust's standard library. We will look at the details of three stages:

\begin{itemize}
    \item First, a methodology for collecting reliable measurements.
    \item Then identifying directions for performance comparisons.
    \item Finally, defining benchmarks.
\end{itemize}

To begin with, I will present a notion of the framework for benchmarking, before diving into details of specific profiling tests.

\section{Methodology}
Big \bigochar{} is a useful tool for reasoning about scalability and performance in theory, and by design, it does not consider factors such as execution environment. First, it disregards constant factors as they are not significant for the growth rate of functions. Second, it does not consider the architecture of CPU and memory \cite{what-programmer-should-know-about-memory}, which indeed influences performance. Furthermore, it also applies to software, such as operating systems, schedulers, virtual machines, et cetera. Hence, often algorithms, which are expected to be equally fast based on \bigochar{}, may differ substantially in performance when evaluated experimentally.

This leads to a need for an experimental performance analysis approach, which would involve executing tests on the actual hardware and software. This, however, introduces another set of unique challenges. For instance, depending on the workload, operating systems may allocate more resources for high demanding tasks, by reducing runtime for others \cite{statistically-rigorous-java-performance-evaluation}. Such non-deterministic behavior may lead to profiling results that vary from run to run significantly.

Benchmarking frameworks were introduced to solve this problem. They are designed to get stable measurements by executing the same test thousands of times. Some of them, such as criterion for Haskell\footnote{\url{https://hackage.haskell.org/package/criterion}} and Rust\footnote{\url{https://crates.io/crates/criterion}}, JMH for Java\footnote{\url{https://openjdk.java.net/projects/code-tools/jmh/}}, and ScalaMeter for Scala\footnote{\url{https://scalameter.github.io/}}, introduce statistical methods for the detection and elimination of exceptionally different runs, known as outliers.

\subsection{Benchmarking frameworks for Rust}
There are several benchmarking frameworks available for Rust, and unfortunately, none of them reached a stable release yet. However, some of them are being actively used in the Rust community and proven to produce reliable results.

There are several criteria which a suitable framework has to meet:
\begin{itemize}
    \item Collecting multiple samples where each sample consists of multiple runs to ensure consistent results.
    \item Detection and elimination of outliers.
    \item A way for setting up a benchmark before each run.
    \item A way for preventing compiler optimizing benchmark code away.
    \item An option to measure execution time without \emph{drop}\footnote{Used to run some code when a value goes out of scope. In other languages, it is sometimes referred to as a destructor.}.
\end{itemize}

\subsubsection*{Rust's benchmark tests}
Rust's testing framework provides an experimental feature\footnote{At the moment of writing, benchmarks are available only in the nightly build of Rust.} that enables developers to write test benchmarks. Those benchmarks are executed thousands of times until results are stabilized. Also, it provides a black-box function\footnote{Black box function contains inline assembly instructions, which compiler cannot make any assumptions about. Hence, it prevents the compiler from optimizing the code, which otherwise would be considered "dead" or unused.} which is opaque for the compiler.

However, it does not detect and eliminate anomalies. It also does not provide APIs for setup routines, which makes it impossible to create benchmarks that rely on certain preconditions.

\subsubsection*{Criterion for Rust}
Criterion for Rust is a powerful and statistically rigorous tool for profiling code. It features outlier elimination, set up routines, and is capable of generating graphs using gnuplot\footnote{\url{http://www.gnuplot.info/}}. At the moment of writing, it is the only framework that has an option to avoid the timing of \emph{drop}.

It is compatible with the stable release of the Rust compiler. Thus, Criterion was chosen as a benchmarking framework for this project.

\subsubsection*{Alternatives}
Other less popular frameworks, such as \emph{bencher}\footnote{\url{https://crates.io/crates/bencher}} and \emph{easybench}\footnote{\url{https://crates.io/crates/easybench}}, were not considered due to the lack of features necessary for the experiment, such as setup routines, optional measurement of \emph{drop} outlier detection, et cetera.

\subsection{Configuration and input size}
All tree-based vector implementations, such as \rbvec{}, \rrbvec{}, \pvec{}, and \imrsvec{}, were configured to use a branching factor of 32, as it provides the best tradeoff for the performance of both read and write operations \cite{efficient-immutable-vectors}.

Each benchmark was executed against a range of input arguments. The input range is specified on the case by case basis depending on the benchmark.

For each input argument, criterion captures 100 samples. The count of runs per sample is determined dynamically by the library to achieve optimal execution time.

Benchmarks of the core operations are executed sequentially on a single thread.

\subsection{Garbage collection}
One of the design goals for this project was to avoid using experimental and unsafe Rust language features. Hence, \pvecrs{}, relies on the memory management means provided by the Rust's standard library only, such as \rc{}. Other automatic memory management mechanisms such as third-party garbage collectors were left out as the future work.

\subsection{Execution environment}
All benchmarks were executed on a computer with an octa-core processor with hyper-threading support, 32GB of DDR4 RAM, and 1TB solid-state drive. The operating system is macOS Catalina 10.15.4, with the stable Rust compiler version 1.41.1.

\begin{table}[!htbp]
    \centering

    \begin{tabular} { |l| p{11cm} | }
        \hline CPU & 2,3 GHz 8-Core Intel Core i9, 16 threads. \\ \hline
        RAM & 32GB DDR4, 2400MHz. \\ \hline
        Disk & 1TB SSD. \\ \hline
        OS & macOS Catalina v10.15.4. \\ \hline
        Rust & v1.41.1. \\ \hline
        Criterion & v0.3.1 \\ \hline
        im-rs & v14.0.0. \\ \hline
    \end{tabular}

    \label{tab:exec-environment}
    \caption{Hardware and software specification used for benchmarking.}
\end{table}

\section{Evaluation directions}
To understand how effective proposed optimizations are, we need to evaluate various tree-based vectors, such as \rbvec{}, \rrbvec{}, and \pvec{}. Implementations from \imrsvec{} and the standard library will be evaluated as well. All vector variants are specified in table \ref{tab:vec-implementations} below.

\begin{table}[!htbp]
    \centering

    \begin{tabular} { |l| p{11cm} | }
        \hline
        \stdvec{} & A vector from the Rust's standard library. \\ \hline
        \rbvec{} & \rbtree{} based vector. \\ \hline
        \rrbvec{} & \rrbtree{} based vector. \\ \hline
        \pvec{} & \rrbtree{} based vector with dynamic internal representation. \\ \hline
        \imrsvec{} & \rrbtree{} based vector from the third party library \emph{im-rs}\footnotemark{}. \\ \hline
    \end{tabular}

    \label{tab:vec-implementations}
    \caption{A table of vector implementations.}
\end{table}

\footnotetext{Both \imrsvec{} and \pvec{} use \rrbtree{} at its core. It has been developed independently in parallel to \pvec{} at the time of writing this paper: \url{https://crates.io/crates/im}.}

The benchmarks described below are categorized into two groups:
\begin{itemize}
    \item Sequential benchmarks for core operations executed on a single thread. They will be executed both against \rc{} and \arc{} variants of the vector.
    \item Parallel benchmarks that are executed against \arc{} based vector only. The goal is to verify whether there are benefits of fast split and combine operations of \rrbvec{}.
\end{itemize}

\subsection{Balanced and relaxed vectors}
As an instance of \rrbtree{} is not perfectly balanced and involves the use of size tables for the radix search, it is expected to be somewhat slower in all core operations. This, however, is not true from the perspective of asymptotic analysis, where constant factors are neglected. The goal of the benchmarks, in this case, is to reveal the overhead induced by the relaxed nodes.

Before each benchmark run, an instance of \rrbvec{} will be prepared by concatenating small vectors together. The count of the relaxed nodes is partially affected by the size of the vector.

\todo{Decide the name of this optimization and use it consistently. Also, it has to be defined somewhere else. }
\subsection{Unique access or transience}
While \rrbvec{} performs very well as a persistent data structure, it is not very optimal when persistence is not required. An example is a function that creates and returns an instance of \rrbvec{}, where all versions except the returned one are disregarded.

Luckily, the persistent vector presented in this project takes advantage of Rust's compiler capabilities of tracking object aliasing. Thus, it avoids redundant copying on mutation if the given object is uniquely accessed. This behavior is similar to transience in the Clojure's persistent vector, but not entirely identical \cite{improving-performance-through-transience}.

In Rust, non-transient, persistent behavior can be enforced by cloning the object before performing a mutation. The objective is to measure the cost of using the clone operation in the persistent vector.

% Rust's compiler guarantees that mutable references are unique, which supposdely eliminates the possibility of a race condition. Essentially, it means that we can assume that it is safe to modify a vector, as we are the only ones who have a mutable access to it.

% By leveraging this assumption, we can make persistent vector temporarily mutable or emphemeral. This comes with performance benefits, as there is no need to copy or clone the vector after each modification anymore. This behavior is somewhat similar to the transient behavior of the Clojure's persistent vector, even though not entirely the same.

\todo{Update the description of dynamic internal representation}
\subsection{Dynamic internal representation}
As one of the suggested optimizations in the paper on \rrbvec{} implementation for Scala \cite{rrb-vector-practical-general-purpose-im-sequence}, a standard vector can be used to improve the performance of small-sized \rrbvec{}. The size for using the standard vector representation is 1024 for the branching factor of 32, and 256 for the branching factor of 4. However, dynamically switching representation during runtime comes at a cost, which potentially may offset the benefits.

The purpose of profiling this optimization is to understand whether it improves performance in practice, and in which use cases. The problem size range will include small values as well.

\subsection{Rc vs Arc}
Atomic reference-counted pointers are expected to introduce additional overhead due to the thread-safety guarantees. The goal is to measure the impact of this overhead in tests, by running the sequential benchmarks first using \rc{}, and then \arc{} pointers.

Parallel benchmarks will be evaluated using only \arc{} pointers, as Rust's compiler forbids passing non-threadsafe types between threads, such as \rc{}.

\section{Sequential benchmarks}
Each benchmark described in this section focuses on a particular core operation of a vector. To avoid ambiguous results, each test exercises only one operation at a time. Operations that modify vector, such as push, will have a complementary version of the benchmark, which also uses the clone operation. This is necessary for comparison of the path copying and naive algorithms used in the tree-based and standard vectors correspondingly.

\todo{Put this table where you first define core operations}
The following operations were evaluated for vector implementations in \ref{tab:vec-implementations}:
\begin{table}[!htbp]
    \centering

    \begin{tabular} { |l| p{10cm} | }
        \hline
        Indexing & Accessing vector values. \\ \hline
        Updating & Updating existing values. \\ \hline
        Pushing & Adding new values to the end of a vector. \\ \hline
        Popping & Removing values at the end of a vector. \\ \hline
        Appending & Concatenating values of one vector to another. \\ \hline
        Splitting & Slicing one vector into two at a given position. \\ \hline
    \end{tabular}

    \label{tab:vec-core-operations}
    \caption{A table of core operations.}
\end{table}

\paragraph{Benchmark structure}
Some benchmarks depend on preconditions. For example, to test indexing, we first need to create a vector with values. Since building a vector instance is not a part of that test, it happens in the setup routine. Hence, benchmarks with preconditions are executed in two steps: setup and the actual test.

\paragraph{Benchmarking dimensions}
Every benchmark for a core operation is parameterized over the vector size. By providing different arguments, we can observe how the performance of vectors is affected in response. This is especially insightful for the tree-based implementations, where the size of the vector influences the height of the tree, which has a negative impact on performance. The output of a benchmark for a given size is the mean runtime in \millis{}.

\subsection{Indexing}
In this section, we will define benchmarks for accessing values in three different configurations, which model the most common ways of working with vector:

\begin{itemize}
    \item Sequentially accessing values by index and iterator.
    \item Accessing values at randomly generated indices.
\end{itemize}

Use cases listed above address two objectives: first, how much overhead relaxed nodes of \rrbtree{} introduce in comparison to \rbtree{}, and second, the efficiency of the dynamic representation in \pvec{}.

The indexing benchmarks share the same setup routine for generating a vector. Balanced tree-based vectors are created by pushing 64-bit integers, while the relaxed types are generated by concatenating vectors together.

The vector size is passed as an argument and falls into the range of \range{[20, \mega{1}]}.

\subsubsection*{Index sequentially and iterating}
The benchmark with access by index loops over the array of \range{[0, N)} indices, and reads values from a vector at corresponding positions. Values are read by using immutable references without taking ownership of them.

The iterators test reads the contents of the tree-based vectors by chunks, rather than by individual values. Additionally, iterator takes ownership of values instead of borrowing them. Due to these differences, the results of this benchmark will not be compared to the access by index.

\subsubsection*{Index randomly}
In this benchmark, values will be read at random positions. Thus, it is quite likely that desired values will be located far apart in memory, potentially causing a cache invalidation. Additionally, results will show whether the performance degenerates with randomness, as it would with linked lists, for example.

By iterating \n{} times, a value is accessed at random index, which is generated within the \range{[0, N)} range by using the \crate{rand} crate\footnote{A Rust library for random number generation: \url{https://crates.io/crates/rand}}. According to the \crate{rand} documentation, generated indices are uniformly distributed. The number generator is explicitly seeded to produce the same stream of randomness between runs.

\subsection{Updating}
There are two dimensions in which the update operation will be evaluated. The first one, similar to the index operation, is the order in which vector values are updated: sequential and random. The second dimension introduces the clone operation used to reveal the cost of copying.

The benchmark list:
\begin{itemize}
    \item Sequentially, with and without \emph{clone}
    \item At random positions, with and without \emph{clone}
\end{itemize}

The setup routine for all benchmarks is identical. As for the index benchmarks, it generates both balanced and relaxed variants of the tree-based vectors. The type of inserted values is an unsigned 64-bit integer.

The vector size is determined by the benchmark argument. The problem size domain for tests using clone is \range{[20, \kilo{20}]}, which is smaller compared to the \range{[20, \kilo{100}]} range, used for benchmarks without clone. This is done to reduce the runtime of benchmarks.

\paragraph{The cost of naive clone vs. path copying}
One of the claimed advantages of \rbvec{} over \stdvec{}, is the cheap clone operation enabled by the path copying algorithm of \rbtree{}. \pvec{} takes advantage of that by switching from the flat to the tree-based representation when cloned. Hence, the objectives are:
\begin{itemize}
    \item Compare performance of naive and path copying algorithms.
    \item Evaluate the efficiency of dynamic representation in \pvec{}.
\end{itemize}

\paragraph{The overhead of relaxed nodes in \rrbtree{}}
Relaxed nodes of \rrbtree{} use size tables to keep track of the size of its child nodes. Balanced nodes, on the other hand, do not need them, as the size can be derived from the node level. Thus, relaxed nodes are more expensive to clone. Additionally, \rrbtree{} is not perfectly balanced as \rbtree{}, potentially resulting in taller trees. The results will reveal how significant this overhead is in practice.

\subsubsection*{Update sequentially}
The test function iterates over indices in the \range{[0, N)} range, where \n{} is the problem size, acquiring a mutable reference to the value at the given position. Once the reference is acquired, it is used to increment the value.

\subsubsection*{Update randomly}
The test function contains a loop, which is executed \n{} times. In the loop body, value is updated by incrementing it, at the index that is randomly generated in the \range{[0, N)} range.

\subsubsection*{Extending benchmarks with the clone operation}
The test with clone introduces an additional variable for keeping track of the cloned vector. This is done to ensure that at least two vector instances exist simultaneously when the update is executed. This is necessary because \rc{} pointers used to implement \rbtree{}, clone the underlying value on mutation only when the reference count is bigger than one. Thus, by having a cloned instance of vector present in the scope, we enforce the path copying algorithm to be used when updating a vector.

\subsection{Pushing}
The push operation is evaluated by populating an empty and existing vectors. Both tests are also extended with the clone operation.

\paragraph{The overhead of relaxed nodes in \rrbtree{}}
The push operation is responsible for increasing the vector capacity. While the vector capacity calculation for \rrbtree{} relies on the size tables, for \rbtree{}, it is sufficient to know the level of the node and the branching factor. Additionally, instantiating relaxed nodes implies the allocation of size tables. All these factors combined are expected to make \rrbtree{}'s push slower compared to \rbtree{}. Thus, there is a dedicated benchmark that uses prebuilt, \rrbtree{}, and \rbtree{} based vectors to evaluate the difference.

\paragraph{Building a vector}
As vector is built from scratch in this benchmark, there is no need for a setup routine. The test function runs a loop over the \range{[0, N)} range of indices, and pushes the index as a value into a vector. The problem size range is \range{[20, \mega{1}]}.

\paragraph{Pushing values into an existing vector}
Push operation does not produce relaxed nodes in balanced trees. Hence, there is no way to evaluate the impact of relaxation in the benchmark of building a vector from scratch. Thus, in this benchmark, values are added to an existing vector, where vector can be both balanced and relaxed depending on the setup routine.

Objectives of the benchmark are:
\begin{itemize}
    \item Check the cost of the complex sub-tree capacity and index computation.
    \item Measure the overhead of using size tables when cloning relaxed nodes.
\end{itemize}

The setup routine generates a vector of the fixed size of \n{} and passes it to the test function. The balanced, \rbtree{} based vector is created by pushing values directly into it, while the \rrbtree{} based one, is created by concatenating several vectors together. Once a vector is created, the test function pushes \n{} values onto it.

\paragraph{Extending benchmarks with the clone operation}
To force \pvec{} to switch from \stdvec{} to \rrbvec{}, several preconditions have to be met, including the reference count being bigger than 1. Hence, the benchmarks above are extended to use clone, in the same way as for benchmarks of the update operation. Beyond the evaluation of \pvec{}, the results are expected to reveal how tree-based vectors stack up to \stdvec{}. The input range for benchmarks using clone is \range{[20, \kilo{40}]}.

\subsection{Popping}
The pop operation manages the vector capacity as well as push. For \stdvec{}, it means shrinking the array and copying elements over. For tree-based vectors, it implies de-allocating nodes and reducing the height of the tree when necessary.

The benchmark is divided into two tests, namely pop and pop clone. The first test calls pop continuously in the loop until the vector is emptied, with the problem size range of \range{[20, \kilo{60}]}.

In the second benchmark, each pop operation will be followed by a clone. Both tests include balanced and relaxed vector types, which are prepared in the setup routine. The problem size range is \range{[20, \kilo{40}]}.

\paragraph{The overhead of relaxed nodes in \rrbtree{}}
Lowering the height of \rrbtree{} involves the tree capacity calculation using size tables, that comes at an additional cost. Thus, this test includes both balanced and relaxed variants of vectors.

\subsection{Appending}
The append operation merges contents of one vector into another. One of the advantages of \rrbtree{} is the relatively low cost of append, that is \bigo{(m^2 \cdot log_m(n)}, in comparison to \bigo{max(a,b)} of \stdvec{}. The objective is to confirm this assumption experimentally.

\paragraph{Naive vs. relaxed append algorithm}
The \rbtree{}-based vector uses a naive concatenation algorithm that simply moves values from one vector to another. \rrbtree{}, on the other hand, merges and re-balances two trees, which is faster in theory. Due to the hardware design specifics, this might not be true for all vector sizes. Thus, benchmarks will reveal how different algorithms perform depending on the size of concatenated vectors.

\paragraph{Appending vectors}
The setup routine prepares a collection of vectors, where each consecutive vector is bigger than the previous. The total size of all prepared vectors adds up to the problem size \n{}. The benchmark is parameterized over the vector size, which will be in the \range{[20, \mega{1}]} range.

Each vector is created by a combination of append and push operations. This way \pvec{} and \rrbvec{} will be forced to use \rrbtree{} for internal representation, while \rbvec{} will remain balanced. \stdvec{} remains flat and does not depend on the type of operation used to add values to it.

The benchmark function iterates over generated vectors and appends them into a vector defined as a local variable.

\subsection{Splitting}
The split operation slices a vector into two parts at the given index. The \rrbtree{}'s algorithm theoretically can achieve good performance by avoiding unnecessary copying. However, due to its complexity, it might be outperformed by naive copying for small-sized vectors.

\paragraph{Naive vs. relaxed splitting algorithm}
This benchmark is aimed to compare \rrbvec{}'s slice algorithm to \stdvec{}'s and \rbvec{}'s naive copying.

\paragraph{Splitting vectors}
The test itself anticipates a prepared vector, which is generated in the setup routine. To evaluate both balanced and relaxed variants of vectors, it generates them differently by either using concatenation or simple pushing. The vector sizes are varying in the \range{[128, \kilo{40}]}.

Once a vector is generated, the benchmark function enters the loop with the condition that the vector needs to contain more than 64 elements. In the loop, a vector is split at index 64, the result of which is assigned back to a variable. Essentially, a vector is being truncated at the front by 64 elements, until it is small enough for the loop to exit.

\section{Parallel benchmarks}
One of the claims is that \rrbvec{} is very efficient when it comes to split and concatenate operations. The data parallelism frameworks, such as Rayon\footnote{\url{https://crates.io/crates/rayon}}, Cilk\footnote{\url{http://supertech.lcs.mit.edu/cilk/}}, and Scala's parallel collections\footnote{\url{https://docs.scala-lang.org/overviews/parallel-collections/overview.html}}, split the work into smaller chunks to ensure good parallelism. Thus, fast split and concatenation operations are critical for optimal performance.

In this section, we will first take a look at how Rayon splits and distributes the work across threads, as well as available configuration parameters. Section \ref{sec:par-benchmarks} introduces tests for benchmarking the overall performance of persistent and standard vectors:

\begin{itemize}
    \item Adding elements of two vectors.
    \item Check if a word is a palindrome.
\end{itemize}

All the tests will be executed on 1, 2, 4, 8, and 16 threads.

Unlike the measurements presented from the sequential benchmarks, the parallel ones include the runtime of both vector operations and Rayon. As the objective is the overall performance comparison, this is considered to be an acceptable tradeoff.

The results will be used to evaluate the effectiveness of the following optimizations:
\begin{itemize}
    \item The effect of relaxed concatenation and split operations of \rrbvec{} on the overall performance.
    \item Dynamic internal representation of \pvec{}.
\end{itemize}

\subsection{Rayon}
Rayon, a data parallelism library for Rust, helps to turn sequential code into parallel with as little work as possible. Loops and iterators are often used to process collections sequentially. Rayon, on the other hand, offers a potentially more efficient alternative to them in the form of parallel iterators. It takes advantage of modern processors, by dividing the work between available cores as long as it is beneficial.

\paragraph{Parallel iterators}
\begin{figure}[!htbp]
    \centering

    \begin{minted}{rust}
        // sequential iterator
        vec![1, 2, 3]
            .into_iter()
            .for_each(|x| println!("{}", x));

        // rayon's parallel iterator
        vec![1, 2, 3]
            .into_par_iter()
            .for_each(|x| println!("{}", x));
    \end{minted}

    \caption{Example of using sequential and parallel iterators.}
    \label{fig:par-iter-example}
\end{figure}

The convenience of Rust iterators is in the provided operators that are called \emph{combinators}. Combinators can be chained and combined, allowing a developer to perform complex manipulations of iterators safely and efficiently.

Parallel iterators provide a similar set of combinators, even though not entirely identical. As iterators process values sequentially, there is a set of combinators that expect values to be emitted in a particular order. As the parallel iterators are designed to process data in any order, inherently sequential combinators are simply not applicable. Thus, Rayon might be not a suitable solution for algorithms relying on the sequential order of execution.

Another limitation which parallel iterators impose, is that type of values which it works with have to implement the \emph{Send} trait. It means using non-threadsafe types such as \rc{} in combination with Rayon is prohibited.

\paragraph{Work splitting}
\begin{figure}[!htbp]
    \centering

    \begin{tikzpicture}[
        font=\ttfamily,
        array/.style={
            matrix of nodes,
            nodes={draw, minimum size=7mm, fill=green!30},
            column sep=-\pgflinewidth,
            row sep=0.5mm,
            nodes in empty cells,
            row 1 column 1/.style={nodes={draw}}
        }]

        \matrix[array] (array) {
            1 & 2 & 3 & 4 & 5 & 6 & 7 & 8 & 9 & 10 & 11 & 12 \\
        };

        \draw[|-|]([yshift=-4mm,xshift=1mm]array-1-1.south west) -- node[above,font=\tiny,outer sep=0mm] {12} ([yshift=-4mm,xshift=-1mm]array-1-12.south east);

        \draw[|-|]([yshift=-8mm,xshift=1mm]array-1-1.south west) -- node[above,font=\tiny,outer sep=0mm] {6} ([yshift=-8mm,xshift=-1mm]array-1-6.south east);
        \draw[|-|]([yshift=-8mm,xshift=1mm]array-1-7.south west) -- node[above,font=\tiny,outer sep=0mm] {6} ([yshift=-8mm,xshift=-1mm]array-1-12.south east);

        \draw[|-|]([yshift=-12mm,xshift=1mm]array-1-1.south west) -- node[above,font=\tiny,outer sep=0mm] {3} ([yshift=-12mm,xshift=-1mm]array-1-3.south east);
        \draw[|-|]([yshift=-12mm,xshift=1mm]array-1-4.south west) -- node[above,font=\tiny,outer sep=0mm] {3} ([yshift=-12mm,xshift=-1mm]array-1-6.south east);
        \draw[loosely dotted]([yshift=-12mm,xshift=1mm]array-1-7.south west) -- ([yshift=-12mm,xshift=-1mm]array-1-12.south east);

        \draw[|-|]([yshift=-16mm,xshift=1mm]array-1-1.south west) -- node[above,font=\tiny,outer sep=0mm] {2} ([yshift=-16mm,xshift=-1mm]array-1-2.south east);
        \draw[|-|]([yshift=-16mm,xshift=1mm]array-1-3.south west) -- node[above,font=\tiny,outer sep=0mm] {1} ([yshift=-16mm,xshift=-1mm]array-1-3.south east);
        \draw[loosely dotted]([yshift=-16mm,xshift=1mm]array-1-4.south west) -- ([yshift=-16mm,xshift=-1mm]array-1-12.south east);

        \draw[|-|]([yshift=-20mm,xshift=1mm]array-1-1.south west) -- node[above,font=\tiny,outer sep=0mm] {1} ([yshift=-20mm,xshift=-1mm]array-1-1.south east);
        \draw[|-|]([yshift=-20mm,xshift=1mm]array-1-2.south west) -- node[above,font=\tiny,outer sep=0mm] {1} ([yshift=-20mm,xshift=-1mm]array-1-2.south east);
        \draw[loosely dotted]([yshift=-20mm,xshift=1mm]array-1-2.south west) -- ([yshift=-20mm,xshift=-1mm]array-1-12.south east);

        \draw ([xshift=1mm]array-1-12.east)--++(0:3mm) node[right]{ Vector };
    \end{tikzpicture}

    \caption{Visualization of work splitting in Rayon.}
    \label{fig:rayon-work-splitting}
\end{figure}

One of the Rayon's components, a fork/join framework, is responsible for dividing and distributing the work between threads. When parallel iterator receives values from a collection like a vector, Rayon attempts to repeatedly divide the work into chunks among threads until the chunk is small enough for a single thread. For an example, see figure \ref{fig:rayon-work-splitting}.

As demonstrated in figure \ref{fig:rayon-join}, the work is \emph{potentially} divided between two threads by calling \mintinline{rust}{rayon::join}, which accepts two closures. Rayon decides whether it is beneficial to parallelize the work, depending on the count of available threads, the split factor, and the workload. If the problem is small enough, it is solved sequentially. Otherwise, it is subdivided into smaller parts. When both closures finish working, the results are combined and returned to the caller.

The size of the work chunk, or the \emph{split factor}, can be controlled by two operations available for \emph{IndexedParallelIterator}, namely \emph{with\_min\_len} and \emph{with\_max\_len}. The use of combinators which may affect the size of a collection, such as \emph{filter}, returns a \emph{ParallelIterator} which does not support configuration of the \emph{split factor}.

\begin{figure}[!htbp]
    \centering

    \begin{minted}{rust}
        rayon::join(
            || do_something(...),
            || do_something_else(...)
        );
    \end{minted}

    \caption{An example for using rayon's join.}
    \label{fig:rayon-join}
\end{figure}

By default, the count of threads allocated by Rayon is equal to the number of cores available in the system. To observe how the thread count affects the performance, Rayon's thread pool will be configured to work with 2, 4, 8, and 16 threads.

\paragraph{Load balancing}
In a perfect world, the chunks of work split between threads take the same amount of time to process. In reality, this is often not the case, resulting in some threads idling. In Rayon, each thread has a queue of work attached to it. It keeps processing the queue until it becomes empty. To avoid idling, the thread which has finished processing its queue can steal work from another thread. This technique is known as work-stealing and is used as the main mechanism for work distribution in Rayon.

\paragraph{Computation stages}
Computation stages of both "Add elements of two vectors" and "Check if a word is a palindrome" benchmarks, can be described in three steps.
\begin{itemize}
    \item First, split the work between threads.
    \item Then process the chunk of work sequentially.
    \item Finally, combine and return the results.
\end{itemize}

The final step can be subdivided further:
\begin{itemize}
    \item Collect individual items into a vector using the parallel \emph{Fold} combinator.
    \item Reduce emitted vectors into a single one using the \emph{Reduce} combinator.
\end{itemize}

\begin{figure}[!htbp]
    \centering

    \begin{minted}{rust}
        let result = parallel_iterator
            .fold(Vec::new, |mut vec, x| {
                vec.push(x);
                vec
            })
            .reduce(Vec::new, |mut vec1, mut vec2| {
                vec1.append(&mut vec2);
                vec1
            });
    \end{minted}

    \caption{Collecting items of parallel iterator.}
    \label{fig:fold-reduce}
\end{figure}

\subsection{Benchmarks}
\label{sec:par-benchmarks}
The benchmarks were executed against following vector implementations: \stdvec{}, \rbvec{}, \rrbvec{}, and \pvec{}, where \rbvec{}, \rrbvec{}, and \pvec{} are based on the threadsafe reference-counted pointer -- \arc{}. The \imrsvec{} is not included because it does not implement the Rayon's \emph{IntoParallelIterator} trait, which makes its evaluation irrelevant.

Benchmarks have been parameterized over two dimensions: the vector size and the number of threads. To see whether parallelism is beneficial, each benchmark has an analogous, sequential implementation executed on a single thread.

\subsubsection*{Sum of elements of two vectors}
Given two equally sized vectors of integers, the test function adds values at the corresponding indices and returns a new instance of a vector with results. The benchmark is subdivided into three steps:

\begin{enumerate}
    \item Transform each vector into a parallel iterator and merge them into a single sequence of value pairs.
    \item Add items of the emitted tuple of two integers into a single value.
    \item Reduce individual sums into a vector of results.
\end{enumerate}

The setup routine prepares two vectors of integers the \range{[0, N]} problem size range.

\subsubsection*{Check if a word is a palindrome}
The benchmark checks whether a word is a palindrome. As input, we are using a list of English words consisting of only alphabetic characters. The benchmark consists of two variants:
\begin{itemize}
    \item Annotating every word with a boolean which indicates if the word is a palindrome.
    \item Collecting palindromes into a new vector.
\end{itemize}

The computation stages of both tests are very similar code-wise, with the difference present in the operators used to process each word:

\begin{enumerate}
    \item Transform the given vector of words into a parallel iterator.
    \item A word processing step for two tests correspondingly:
    \begin{enumerate}
        \item Return a tuple containing the word and the flag which indicates whether the word is a palindrome.
        \item Filter out all words which are not palindromes.
    \end{enumerate}
    \item Reduce the results to a new instance of a vector.
\end{enumerate}

Essentially, the difference comes down to the operators used. For the first test, each word is processed by the \emph{Map} combinator. It takes a string as an argument and returns a tuple of a string and boolean.

The second test relies on the \emph{Filter} combinator, which takes a predicate as an argument. The predicate, in this case, is the function checking if the word is a palindrome.

An important difference in behavior between two tests is that filter alters the length of the resulting vector, while the map combinator does not.

Hence, the difference in performance between standard and persistent vector might become more apparent in the second test, as Rayon will not be able to make optimizations based on the assumptions made about the size of the resulting vector.

\paragraph{The benchmark setup}
The total count of words stored in the file is 370103. As the benchmark is parameterized over the size of a vector, which in this case is equal to the number of words, the setup routine caps the count of words by \n{}.

The contents of the file are loaded into memory once, and then before each run, they are copied over to a new vector within the setup routine. A vector is later passed to the test routine that converts it to a parallel iterator.

\section{Measuring the memory footprint}
As the space complexity is as important as time, this section presents a methodology used to evaluate the memory footprint of both the tree-based and flat vector types.

\subsection{Requirements}

Here are the things that we expect from the memory benchmark suite:
\begin{itemize}
    \item The overall memory footprint of the vector, including the use of the stack and the heap memory.
    \item Reliable and reproducible results.
\end{itemize}

\subsection{Memory benchmarking tools}
As of the time of writing, there is no established practice or a framework for measuring the memory usage of the Rust libraries.

However, the operating systems already come with the profiling tools for measuring memory usage. These tools should provide reliable results, as the memory footprint of the process is not changing as much between runs. Hence, it should be sufficient to run each test once.

Two approaches were evaluated:
\begin{itemize}
    \item Measuring heap allocations by printing statistics from the system allocator.
    \item Capturing the memory usage of the process when running the benchmark application.
\end{itemize}

\subsubsection*{Allocation tracking}
The first option is not ideal because it only evaluates the heap memory usage. Additionally, a custom memory allocator such as \crate{jemallocator}\footnote{\url{https://crates.io/crates/jemallocator}} is required to retrieve statistics. Using a non-default allocator in tests also means that benchmarks will not accurately reflect the performance in production applications.

\subsubsection*{Process benchmarking}
Benchmarking by measuring the process footprint allows us to capture both the stack and the heap usage, and it does not depend on any special Rust debug or allocator features. This, in turn, means that the benchmarks can be optimized as if they were compiled for the production use, putting them as close as possible to the real use cases.

\paragraph{Implementation}
A new module is introduced to \pvecrs{} -- \crate{pvec-memory}, that includes the binary for executing the memory benchmarks. Its responsibilities are to configure, run and collect the test results. It will be referred to as \emph{memory bencher} from now on. \todo{A call graph for the memory bencher, time, and benchmarks.}

Memory bencher contains a list of tests, input sizes and the vector types. It executes benchmarks by running them through the CLI program called \emph{time}, which outputs information about the process including the maximum amount of memory used in bytes. Memory bencher parses this value from the \emph{maximum resident set size} field.

The program code occupies a part of the process memory, and needs to be accounted for. The bencher measures the footprint of the program without running any tests, and then subtracts it from the benchmark results.

It is important to emphasize that the \emph{time} utility included in macOS is different from the GNU\footnote{\url{https://www.gnu.org/software/time/}} variant, that comes with more options and outputs the \emph{maximum resident set size} in kilobytes instead of bytes. The memory bencher was only tested on macOS, and requires changes to support other platforms.

\subsection{Benchmarks}
The goals for the memory tests are the following:
\begin{itemize}
    \item Measure and compare the memory footprint of the tree-based and standard vectors.
    \item Evaluate the effectiveness of the structural sharing when cloning a vector.
\end{itemize}

\subsubsection*{Building a vector}
This benchmark is expected to reveal the memory overhead of using a tree instead of the contiguous memory block as its height, and the node count grows. Given the size \n{} as a parameter, this benchmark builds a vector by pushing \n{} values into it. The problem size range is \range{[20, 60k]}.

\subsubsection*{Updating and cloning a vector}
The test builds a new vector of size \n{}, runs a loop from 0 to \n{}, clones a vector, and updates a value at the given index. All cloned instances are accumulated in another vector to observe how well structural sharing helps to save memory. The vector sizes are in \range{[20, 60k]} range.

\section{Presentation of results}
The measurements are done in a number of samples that can be configured. By default, the count of samples is set to a 100, and this default is being used for this project. Each sample consists of one or typically more iterations of the routine. The elapsed time between the beginning and the end of the iterations, divided by the number of iterations, gives an estimate of the time taken by each iteration.

As measurement progresses, the sample iteration counts are increased. The presented graphs use the mean measured time for each function. Additionally, benchmarks were configured to avoid timing the execution of \emph{drop}.

\section{Reproducing results}
Benchmarks can be compiled and executed using Rust's package manager named \emph{cargo}. By default, sequential benchmarks are executed against the non-threadsafe variant of the persistent vector. To select the threadsafe variant, the user can pass the \arc{} feature flag as demonstrated in figure \ref{fig:sequential-benches}. To run a particular benchmark, specify its name as an argument to cargo.

\floatstyle{boxed}
\begin{figure}[!htbp]
    \centering

    \begin{minted}{bash}
        # benches of the non-threadsafe implementation
        cargo bench

        # benches of the threadsafe implementation
        cargo bench --features=arc
    \end{minted}

    \caption{Running sequential benchmarks.}
    \label{fig:sequential-benches}
\end{figure}

To execute parallel benchmarks, user needs to pass both the \arc{} and \emph{rayon-iter} feature flags:

\begin{figure}[!htbp]
    \centering

    \begin{minted}{bash}
        cargo bench --features=arc,rayon-iter
    \end{minted}

    \caption{Running parallel benchmarks.}
    \label{fig:parallel-benches}
\end{figure}

Criterion can also generate HTML reports with charts using \emph{gnuplot}\footnote{\url{http://www.gnuplot.info/}}. Results can be found in the \emph{pvec-rs/target/criterion} directory, and if \emph{gnuplot} is available, the report will be located at \emph{pvec-rs/target/criterion/report/index.html}.

For more information on available options and parameters for Criterion, please consult the library documentation\footnote{\url{https://docs.rs/criterion/0.3.1/criterion/}}.

\subsection{Memory benchmark results}
Memory benchmarks produce consistent and reproducible results. Hence it is enough to execute them only once. Due to the differences in how \emph{time} works on different operating systems, macOS is the only supported platform at the moment. Figure \ref{fig:memory-benches} demonstrates how to execute the tests:

\floatstyle{boxed}
\begin{figure}[!htbp]
    \centering

    \begin{minted}{bash}
        # navigate to the pvec-memory crate
        cd pvec-memory

        # execute the script that invokes
        # cargo to compile and run benchmarks
        # in the release mode
        sh benches.sh
    \end{minted}

    \caption{Running memory benchmarks.}
    \label{fig:memory-benches}
\end{figure}

The generated report is located in the \emph{pvec-rs/target/release/report} directory, and it is subdivided into folders, each named after the benchmark. Each folder contains csv\footnote{\url{https://tools.ietf.org/html/rfc4180}} files, where each file is name after the evaluated vector type. The report file contains two columns corresponding to the input size and the memory footprint in bytes.
