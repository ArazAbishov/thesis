% ToDo:
%  - Reconsider if specifying the sample count and count of runs per sample is useful. 
%  - Fill-in more details on how benchmarks are analyzed from Criterion's documentation
%  - Re-introduce what is a black box, and update benchmarks to avoid misusing it. 
%  - You need to be careful with using next words interchangeably: profile, measure, benchmark
%  - What bootstrap sampling means?
%  - What kind of benchmarks are you presenting? Throughput or latency?

% Notes from the criterion documentation: 
% - Note from Criterion: note that Criterion.rs does not measure each individual iteration, only the complete sample. The resulting samples are stored for use in later stages.
% - These two timing loops generate a batch of inputs and measure the time to execute the benchmark on all values in the batch. As with iter_with_large_drop they also collect the values returned from the benchmark into a Vec and drop it later without timing the drop. 
% - Keep in mind that this is only necessary if the benchmark modifies the input - if the input is constant then one input value can be reused and the benchmark should use iter instead.
% - You should write about what is drop in Rust, and how it is relevant in Criterion benchmarks (whether it is being measured as well or not). 

\chapter{Results and discussion}
\section{Performance of core operations}

\subsection{Indexing}

In this section we will look at the performance of the index operation in practice. The evaluation will include two common use cases, such as accessing elements sequentially and randomly. % ToDo: add more points here later

\subsubsection*{Index Sequentially}

Accessing elements sequentially is a common way of reading values from a vector. The following benchmark is dedicated to understanding the performance properties of \rbtree{} and \rrbtree{} based vector implementations in comparison to the vector from the Rust's standard library. 

As the size influences the internal representation of a vector, it is critical to ensure that it performs well consistently over a wide range of inputs. The results will reveal whether the size based optimization is effective, and how sigfnicant the impact of the increasing height of \rbtree{} and \rrbtree{} is. 

The problem size interval is [10, 1M]\todo{typesetting?}, with a number of values in between. The average run time for each given problem size is recorded as the result in \ref{fig:index-sequentially}. 

The benchmark consists of two parts: the setup routine and the actual test function. The setup routine, however, differs for the balanced and unbalanced flavors of vectors. 

The balanced vector is created by pushing elements \footnote{The type of elements is a 64-bit unsigned integer.} into it, while an unbalanced one is created by concatenating several small vectors together. The test function loops over the [0, N] interval retrieving corresponding values from a vector, where N \todo{typesetting?} is the problem size. 

\begin{figure}        
    \caption{Benchmarking results of indexing sequentially.}
    \label{fig:index-sequentially}

    \begin{tikzpicture} 
        \pgfplotstableread[col sep=comma]{chapters/data/index_sequentially/im-rs-rbtree.csv}\imrsrbtree;
        \pgfplotstableread[col sep=comma]{chapters/data/index_sequentially/im-rs-rrbtree.csv}\imrsrrbtree;
        \pgfplotstableread[col sep=comma]{chapters/data/index_sequentially/pvec-rbtree.csv}\pvecrbtree;
        \pgfplotstableread[col sep=comma]{chapters/data/index_sequentially/pvec-rrbtree.csv}\pvecrrbtree;
        \pgfplotstableread[col sep=comma]{chapters/data/index_sequentially/rbvec.csv}\rbvec;
        \pgfplotstableread[col sep=comma]{chapters/data/index_sequentially/rrbvec.csv}\rrbvec;
        \pgfplotstableread[col sep=comma]{chapters/data/index_sequentially/std.csv}\std;    
        
        % ToDo:
        %  - Axis should present numbers in a more readable form (you can probably specify them yourself too)
        %  - Fonts used for axis and plot legengs (change the fonts, reduce their size)    
    
        \begin{loglogaxis}[         
            use units,         
            y unit=s, y unit prefix=m, 
            smooth,        
            width=340pt, % width=\textwidth, 
            title={Index Sequentially},
            xlabel={vector size (log scale)},
            ylabel={mean time (log scale)},                
            ymajorgrids=true, 
            xmajorgrids=true,
            grid style=dashed,
            legend pos=north west,        
            legend style={draw=none,fill=none,font=\footnotesize},
            legend cell align=left, 
            yticklabel style={
                /pgf/number format/fixed, 
                /pgf/number format/precision=5
            },
            yticklabel={%
                \pgfmathfloatparsenumber{\tick}%
                \pgfmathfloatexp{\pgfmathresult}%
                \pgfmathprintnumber{\pgfmathresult}%
            },
            xticklabel style={/pgf/number format/fixed},		
            xticklabel={%
                \pgfmathfloatparsenumber{\tick}%
                \pgfmathfloatexp{\pgfmathresult}%
                \pgfmathprintnumber{\pgfmathresult}%
            },
        ]
            \addplot[thin, color=morange, mark=*,] table {\std}; 
            \addlegendentry{Vec}
    
            \addplot[thin, color=mred, mark=*,] table {\rbvec};
            \addlegendentry{Balanced RrbVec}
    
            \addplot[thin, color=mred, mark=pentagon,] table {\rrbvec}; 
            \addlegendentry{Unbalanced RrbVec}
    
            \addplot[thin, color=mgreen, mark=square,] table {\pvecrbtree};
            \addlegendentry{Balanced PVec}
    
            \addplot[thin, color=mgreen, mark=diamond*,] table {\pvecrrbtree};
            \addlegendentry{Unbalanced PVec}
    
            \addplot[thin, color=mpurple, mark=pentagon*,] table {\imrsrbtree};
            \addlegendentry{Balanced im-rs Vector}
    
            \addplot[ultra thin, color=mpurple, mark=square,] table {\imrsrrbtree};
            \addlegendentry{Unbalanced im-rs Vector} 
        \end{loglogaxis}     
    \end{tikzpicture} 
\end{figure}

% \begin{figure}        
%     \caption{Benchmarking results of iterating through vector.}
%     \label{fig:iter-next}

%     \begin{tikzpicture} 
%         \pgfplotstableread[col sep=comma]{chapters/data/iterator_next/im-rs-vector-balanced.csv}\imrsrbtree;
%         \pgfplotstableread[col sep=comma]{chapters/data/iterator_next/im-rs-vector-unbalanced.csv}\imrsrrbtree;
%         \pgfplotstableread[col sep=comma]{chapters/data/iterator_next/pvec-balanced.csv}\pvecrbtree;
%         \pgfplotstableread[col sep=comma]{chapters/data/iterator_next/pvec-unbalanced.csv}\pvecrrbtree;
%         \pgfplotstableread[col sep=comma]{chapters/data/iterator_next/rrbvec-balanced.csv}\rbvec;
%         \pgfplotstableread[col sep=comma]{chapters/data/iterator_next/rrbvec-unbalanced.csv}\rrbvec;
%         \pgfplotstableread[col sep=comma]{chapters/data/iterator_next/std-vec.csv}\std;    
        
%         % ToDo:
%         %  - Axis should present numbers in a more readable form (you can probably specify them yourself too)
%         %  - Fonts used for axis and plot legengs (change the fonts, reduce their size)    
    
%         \begin{loglogaxis}[         
%             use units,         
%             y unit=s, y unit prefix=m, 
%             smooth,        
%             width=340pt, % width=\textwidth, 
%             title={Iterator Next},
%             xlabel={vector size (log scale)},
%             ylabel={mean time (log scale)},                
%             ymajorgrids=true, 
%             xmajorgrids=true,
%             grid style=dashed,
%             legend pos=north west,        
%             legend style={draw=none,fill=none,font=\footnotesize},
%             legend cell align=left, 
%             yticklabel style={
%                 /pgf/number format/fixed, 
%                 /pgf/number format/precision=5
%             },
%             yticklabel={%
%                 \pgfmathfloatparsenumber{\tick}%
%                 \pgfmathfloatexp{\pgfmathresult}%
%                 \pgfmathprintnumber{\pgfmathresult}%
%             },
%             xticklabel style={/pgf/number format/fixed},		
%             xticklabel={%
%                 \pgfmathfloatparsenumber{\tick}%
%                 \pgfmathfloatexp{\pgfmathresult}%
%                 \pgfmathprintnumber{\pgfmathresult}%
%             },
%         ]
%             \addplot[thin, color=morange, mark=*,] table {\std}; 
%             \addlegendentry{Vec}
    
%             \addplot[thin, color=mred, mark=*,] table {\rbvec};
%             \addlegendentry{Balanced RrbVec}
    
%             \addplot[thin, color=mred, mark=pentagon,] table {\rrbvec}; 
%             \addlegendentry{Unbalanced RrbVec}
    
%             \addplot[thin, color=mgreen, mark=square,] table {\pvecrbtree};
%             \addlegendentry{Balanced PVec}
    
%             \addplot[thin, color=mgreen, mark=diamond*,] table {\pvecrrbtree};
%             \addlegendentry{Unbalanced PVec}
    
%             \addplot[thin, color=mpurple, mark=pentagon*,] table {\imrsrbtree};
%             \addlegendentry{Balanced im-rs Vector}
    
%             \addplot[ultra thin, color=mpurple, mark=square,] table {\imrsrrbtree};
%             \addlegendentry{Unbalanced im-rs Vector} 
%         \end{loglogaxis}     
%     \end{tikzpicture} 
% \end{figure}

% TODO:
%  - Insert the benchmarks for iterator_next
%  - Role of transience in this benchmark
%  - Rc vs Arc (you will need to sub-merge results of that run into the plot?)
%  - Claimed perf guarantees in big-o notation

Unsurprisingly, the \stdvec{} shows the best results in this test. As it is backed by a contiguous chunk of memory, it takes full advantage of cache locality. In addition, its structure is not affected by the method used to build it, when \rbtree{} and \rrbtree{} based vectors are. 

\paragraph{Dynamic internal representation}

\pvec{} switches its internal representation from the standard vector to \rrbvec{} when the count of elements surpasses a certain threshold, which was set to 4096 in this case. This is evident from the plot \ref{fig:index-sequentially}, where \pvec{} is 3.44x faster than balanced \rrbvec{}, but slower than \stdvec{} by a factor of 1.79x. It is slower because of the overhead induced by acquiring a mutable reference to \rc{} pointer. As soon as the threshold is passed, the difference in performance between \pvec{} and \rrbvec{} becomes negligible. 

% ToDo: should you check error margins, and use them as reference for negligible differences? 

\paragraph{Balanced vs unbalanced variants}
The difference between the balanced and unbalanced \rrbvec{} becomes noticeable as the problem size grows. The balanced variant is faster than the unbalanced one by a factor of 1.68x in the [100, 1M] input range. This is expected, because the unbalanced variant introduces relaxed nodes, which rely on the size tables to compute the path to the value. 

This, however, is not the case for small problem sizes in the [0, 100] range, for which the concatenating algorithm produces a well balanced tree. Hence, both balanced and unbalanced vectors have identical performance numbers in that range. 

Both balanced and unbalanced \imrsvec{} variants tend to be slower in comparison to \rrbvec{} in the [100, 1M] input range] by a factor of 4.1x. However, for smaller inputs it is slightly faster with a difference of 1.18x. 

% ToDo: be more explicit when it comes to comparison to std, because now there is almost no mention do how fast or slow PVec is. 

\paragraph*{Rc vs Arc}
\todo{Run benchmarks for Arc}

\subsubsection*{Index Randomly}
In this benchmark, rather than accessing consecutive indices, we will get elements at random positions. From the algorithmic point of view, the implementation of the operation itself is the same. The difference is mostly in the behavior of processor caches, which will be invalidated if the requested element is out of reach. This is quite likely to happen as indices are picked randomly. In addition, the results will clarify whether the performance degenerates with randomness, as it would with linked lists for example. 

\paragraph*{Setup}
The setup routine is identical to the one in the index sequentially benchmark. The difference is in the actual test routine itself, where the \todo{ref} \emph{rnd} library is used to generate indices randomly. 

\paragraph*{Dynamic internal representation}
It is clear that \pvec{} is as fast as \stdvec{} up to the size of 4096, right after which it becomes slower and aligns with the \rrbvec{}. Even then, however, performance difference between \stdvec{} and \pvec{} is not significant, remaining quite consistent over the rest of the input range at 2.7x. 

\paragraph*{Balanced vs unbalanced variants}
Both variants perform well, with the balanced \rrbvec{} being faster by 1.36x in [100, 1M] range. Both unbalanced \rrbvec{} and \imrsvec{} are equally fast with insignificant marginal differences.

\begin{figure}        
    \caption{Benchmarking results of indexing randomly.}
    \label{fig:index-randomly}

    \begin{tikzpicture} 
        \pgfplotstableread[col sep=comma]{chapters/data/index_randomly/im-vector-balanced.csv}\imrsrbtree;
        \pgfplotstableread[col sep=comma]{chapters/data/index_randomly/im-vector-unbalanced.csv}\imrsrrbtree;
        \pgfplotstableread[col sep=comma]{chapters/data/index_randomly/pvec-balanced.csv}\pvecrbtree;
        \pgfplotstableread[col sep=comma]{chapters/data/index_randomly/pvec-unbalanced.csv}\pvecrrbtree;
        \pgfplotstableread[col sep=comma]{chapters/data/index_randomly/rrbvec-balanced.csv}\rbvec;
        \pgfplotstableread[col sep=comma]{chapters/data/index_randomly/rrbvec-unbalanced.csv}\rrbvec;
        \pgfplotstableread[col sep=comma]{chapters/data/index_randomly/std.csv}\std;    
        
        % ToDo:
        %  - Axis should present numbers in a more readable form (you can probably specify them yourself too)
        %  - Fonts used for axis and plot legengs (change the fonts, reduce their size)    
    
        \begin{loglogaxis}[         
            use units,         
            y unit=s, y unit prefix=m, 
            smooth,        
            width=340pt, % width=\textwidth, 
            title={Index Randomly},
            xlabel={vector size (log scale)},
            ylabel={mean time (log scale)},                
            ymajorgrids=true, 
            xmajorgrids=true,
            grid style=dashed,
            legend pos=north west,        
            legend style={draw=none,fill=none,font=\footnotesize},
            legend cell align=left, 
            yticklabel style={
                /pgf/number format/fixed, 
                /pgf/number format/precision=5
            },
            yticklabel={%
                \pgfmathfloatparsenumber{\tick}%
                \pgfmathfloatexp{\pgfmathresult}%
                \pgfmathprintnumber{\pgfmathresult}%
            },
            xticklabel style={/pgf/number format/fixed},		
            xticklabel={%
                \pgfmathfloatparsenumber{\tick}%
                \pgfmathfloatexp{\pgfmathresult}%
                \pgfmathprintnumber{\pgfmathresult}%
            },
        ]
            \addplot[thin, color=morange, mark=*,] table {\std}; 
            \addlegendentry{Vec}
    
            \addplot[thin, color=mred, mark=*,] table {\rbvec};
            \addlegendentry{Balanced RrbVec}
    
            \addplot[thin, color=mred, mark=pentagon,] table {\rrbvec}; 
            \addlegendentry{Unbalanced RrbVec}
    
            \addplot[thin, color=mgreen, mark=square,] table {\pvecrbtree};
            \addlegendentry{Balanced PVec}
    
            \addplot[thin, color=mgreen, mark=diamond*,] table {\pvecrrbtree};
            \addlegendentry{Unbalanced PVec}
    
            \addplot[thin, color=mpurple, mark=pentagon*,] table {\imrsrbtree};
            \addlegendentry{Balanced im-rs Vector}
    
            \addplot[ultra thin, color=mpurple, mark=square,] table {\imrsrrbtree};
            \addlegendentry{Unbalanced im-rs Vector} 
        \end{loglogaxis}     
    \end{tikzpicture} 
\end{figure}


\subsection{Updating}
\subsection{Pushing}

The push operation allows to add new values at the end of the vector. In this benchmark, we are going to evaluate two use cases:

\begin{itemize}
    \item Building a vector from scratch by pushing values
    \item Pushing values into a predefined, unbalanced vector
\end{itemize}

To evaluate how effective the path copying algorithm is, the use cases above will be extended with a clone operation following every push. 

\begin{itemize}
    \item Pushing values and cloning a balanced vector
    \item Pushing values and cloning an unbalanced vector
\end{itemize}

\subsubsection*{Building a vector}
The focus of the first benchmark is to measure how much time it takes to build a vector of different sizes. In general, the test was designed in a way that vectors cannot make assumptions of their own resulting sizes. Hence, optimizations for pre-allocating the space were not used. 

In the second benchmark, tests are running against unbalanced variant of vector, which is pre-built in the setup routine. The results will show the overhead introduced by the relaxed nodes in the tree. 

\subsection{Popping}
% pop balanced / unbalanced

\subsection{Concatenating}
\subsection{Slicing}

\section{Performance in parallel benchmarks}
\section{Effectivness of optimizations}
\subsection{Overhead of RRB Tree}
\subsection{Unique access or transience}
\subsection{Dynamic internal representation}
\section{Memory footprint}
