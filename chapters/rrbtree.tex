\chapter{Relaxed Radix Balanced Tree}

% A couple of words here on what is RRB-Tree, what it offers, and how is it different compared to RB-Tree.
%  * A confluent persistent data structure.
%  * Relaxation as a way to achieve efficient concatenation and splitting.

% Then, say what this chapter is about:
%  * Which constraints are relaxed and which constraints are enforced:
%   * The relaxed child count and completeness of nodes
%   * How does RRB-Tree guarantee that height won't exceed certain limits
%  * The concatenation and splitting algorithms.
%  * Changing the following algorithms to accommodate relaxed constraints:
%   * Radix search
%   * Push and pop

\section{Relaxed Radix Search}

When \rbtree\ is relaxed, it is no longer possible to compute indices from a search key without additional metadata in the form of \emph{size tables}. Each entry of the size table is the accumulated count of values in the corresponding subtree. A node contains a value if a table entry is bigger than the search key corresponding to it. When a balanced node is encountered, the search process falls back to the original radix search algorithm \Cref{sec:rb-tree-radix-search}.

\begin{listing}[ht!]
    \caption{Pseudocode for the relaxed radix search implementation}
    \label{lst:rrb-tree-relaxed-radix-search}

    \begin{algorithmic}
        \Function{RrbTree-Find-Index}{sizes, idx}
            \State candidate \la 0

            \If{candidate < \m\ - 1 \And\ sizes[candidate] <= idx}
                \State candidate++
            \EndIf

            \State \Return candidate
        \EndFunction

        \State

        \Function{RrbTree-Relaxed-Radix-Search}{root, key}
            \State node \la\ root
            \State idx \la\ key

            \For{level \la\ root\ts{height} - 1, 1}
                \If{node\ts{sizes} = \nil{}}
                    \State index \la\ (key $\ggg$ (level * x)) \& mask
                    \State node \la\ node[index]
                \Else
                    \State sizes \la\ node\ts{sizes}
                    \State index \la\ RrbTree-Find-Index(sizes, idx)
                    \State node \la\ node[index]

                    \If{index != 0}
                        \State idx \la\ idx - sizes[index - 1]
                    \EndIf
                \EndIf
            \EndFor

            \State index \la\ idx \& mask
            \State \Return {node[index]}
        \EndFunction
    \end{algorithmic}
\end{listing}

\section{Concatenation}

\todo{Make sure that nodes are cloned where necessary}
\begin{listing}[ht!]
    \caption{Pseudocode for concatenation implementation}
    \label{lst:rrb-tree-concat}

    \begin{algorithmic}
        \Procedure{Node-Check-Subtree}{root, subtree}
            \If{subtree\ts{len} = \m{}}
                \State root[root\ts{len}] \la\ subtree
                \State root\ts{len}++
            \EndIf
        \EndProcedure

    \end{algorithmic}

    \begin{algorithmic}

        \Function{Node-Rebalance}{left, middle, right}
            \State height \la\ Max(left\ts{height}, middle\ts{height}, right\ts{height})
            \State merged \la\ left + middle + right

            \State newRoot \la\ Create-Node(height)
            \State newSubtree \la\ Create-Node(height)
            \State newNode \la\ Create-Node(height)

            \For{node \In\ merged}
                \If{newNode\ts{len} = 0 \And\ node\ts{len} == \m{}}
                    \State Node-Check-Subtree(newRoot, newSubtree)
                    \State newSubtree[newSubtree\ts[len]] \la\ node
                    \State newSubtree\ts{len}++
                \Else
                    \For{childNode \In\ node}
                        \If{newNode\ts{len} = \m{}}
                            \State Node-Check-Subtree(newRoot, newSubtree)
                            \State newSubtree[newSubtree\ts[len]] \la\ newNode
                            \State newSubtree\ts{len}++
                        \EndIf
                    \EndFor
                \EndIf
            \EndFor

            \State Node-Check-Subtree(newRoot, newSubtree)

            \If{newNode\ts{len} != 0}
                \State newSubtree[newSubtree\ts[len]] \la\ newNode
                \State newSubtree\ts{len}++
            \EndIf

            \If{newSubtree\ts{len} != 0}
                \State newRoot[newRoot\ts[len]] \la\ newSubtree
                \State newRoot\ts{len}++
            \EndIf

            \State \Return newRoot
        \EndFunction

    \end{algorithmic}

    \begin{algorithmic}


        \Function{Node-Concat}{leftNode, rightNode}
            \If{leftNode\ts{height} > rightNode\ts{height}}
                \State mergedNode \la\ Node-Concat(leftNode\ts{last}, rightNode)
                \State \Return Node-Rebalance(leftNode\ts{init}, mergedNode, \nil{})
            \ElsIf{leftNode\ts{height} < rightNode\ts{height}}
                \State mergedNode \la\ Node-Concat(leftNode, rightNode\ts{first})
                \State \Return Node-Rebalance(\nil{}, mergedNode, rightNode\ts{tail})
            \Else
                \State mergedNode \la\ \nil{}

                \If{leftNode\ts{height} = 0}
                    \State mergedNode \la\ Leaf-Node-Concat(leftNode, rightNode)
                \Else
                    \If{leftNode\ts{height} = 1}
                        \State mergedNode \la\ Leaf-Node-Concat(leftNode\ts{last}, rightNode\ts{first})
                    \Else
                        \State mergedNode \la\ Node-Concat(leftNode\ts{last}, rightNode\ts{first})
                    \EndIf
                \EndIf

                \State \Return Node-Rebalance(leftNode\ts{init}, mergedNode, rightNode\ts{tail})
            \EndIf

            \State \Return \nil{}
        \EndFunction

    \end{algorithmic}

    \begin{algorithmic}

        \Function{RrbTree-Concat}{thisRoot, thatRoot}
            \State newRoot \la\ Node-Concat(thisRoot, thatRoot)
            \State newRoot\ts{height} \la\ Max(thisRoot\ts{height}, thatRoot\ts{height})
            \State newRoot\ts{len} \la newRoot\ts{len} + thatRoot\ts{len}

            \If{newRoot\ts{len} = 1}
                \State newRoot \la\ newRoot[0]
            \Else
                \State newRoot\ts{height}++
            \EndIf

            \State \Return \nil{}
        \EndFunction

    \end{algorithmic}
\end{listing}


\begin{listing}[ht!]

    \begin{minted}[
        fontsize=\small,
        stripnl=false,
        framesep=4mm,
        frame=lines,
        autogobble,
        linenos
    ]{rust}

    fn merge(&mut self, mut that: Node<T>,
        self_shift: Shift, that_shift: Shift) -> Node<T> {
        if self_shift > that_shift {
            let (left, init) = self.as_branch().split_last_must();
            let center = left.merge(that, self_shift.dec(), that_shift);
            Node::rebalance(init, center, None, self_shift)
        } else if self_shift < that_shift {
            let (right, tail) = that.as_branch().split_first_mut();
            let center = self.merge(right, self_shift, that_shift.dec());
            Node::rebalance(None, center, tail, that_shift)
        } else {
            let (left, init) = self.as_branch().split_last_must();
            let (right, tail) = that.as_branch().split_first_mut();
            let mut center = if self_shift.is_level_with_leaves() {
                SharedPtr::make_mut(left.as_mut_leaf())
                    .merge(right.into_leaf())
            } else {
                left.merge(right, self_shift.dec(), that_shift.dec())
            };
            Node::rebalance(init, center, tail, self_shift)
        }
    }

    \end{minted}

    \caption{Concatenation algorithm}
    \label{lst:rrb-tree-concat}
\end{listing}

\begin{listing}[ht!]
    \caption{Pseudocode for concatenation implementation}
    \label{lst:rrb-tree-concat}

    \begin{algorithmic}[1]
        \Function{concat}{leftNode, rightNode}
            \If{leftNode\ts{height} > rightNode\ts{height}}
                \State mergedNode \la\ concat(leftNode\ts{last}, rightNode)
                \State \Return rebalance(leftNode\ts{init}, mergedNode, \nil{})
            \ElsIf{leftNode\ts{height} < rightNode\ts{height}}
                \State mergedNode \la\ concat(leftNode, rightNode\ts{first})
                \State \Return \Call{rebalance}{\nil{}, mergedNode, rightNode\ts{tail}}
            \Else
                \State mergedNode \la\ \nil{}

                \If{leftNode\ts{height} = 0}
                    \State mergedNode \la\ concat(leftNode, rightNode)
                \Else
                    \If{leftNode\ts{height} = 1}
                        \State mergedNode \la\ concat(leftNode\ts{last}, rightNode\ts{first})
                    \Else
                        \State mergedNode \la\ concat(leftNode\ts{last}, rightNode\ts{first})
                    \EndIf
                \EndIf

                \State \Return rebalance(leftNode\ts{init}, mergedNode, rightNode\ts{tail})
            \EndIf
        \EndFunction
    \end{algorithmic}

\end{listing}