\newcommand{\boxptr}{\type{box}}

\todo{Use the "thread safety" phrease consistently in the document}
\todo{Define the types that are provided by pvec-rs}
\todo{Define the API and methods that are implemented}
\todo{Reference the software project: c-rrb}

\chapter{Persistent vector in Rust}
\label{chapter:persistent-vector}

This chapter discusses the advantages and disadvantages of different approaches to defining the persistent vector in Rust, as well as their influence on the data structure design:

\begin{itemize}
    \item First, we will take a look at the memory layout of \rrbtree{} and how it supports the implementation of cloning and path copying.
    \item Then, transience and dynamic representation optimizations will be introduced in the context of Rust's ownership and borrowing rules.
    \item At last, we will talk about thread safety and the requirements for sharing objects between threads.
\end{itemize}

\section{Memory layout}
To define the persistent vector in Rust, we first need to understand how different parts of the data structure can be represented in the computer memory, and which Rust constructs are the most suitable for this purpose.

The foundation of a confluently persistent vector is \rrbtree{}, with additional fields such as a tail, size, and height. \rrbtree{} consists of infinitely nested nodes, which form a directed acyclic graph\footnote{A directed acyclic graph is a data structure that consists of nodes connected with directed edges, in which moving from node to node by following edges will never lead to the same node again.} in memory.

As the persistent vector can be arbitrarily large, the Rust compiler is not able to measure its size during compilation. Hence, the memory allocation on the stack is not possible without additional constraints, such as the fixed vector capacity.

In fact, the Rust compiler will abort the compilation if a \emph{recursive data type}\footnote{\url{https://doc.rust-lang.org/book/ch15-01-box.html\#enabling-recursive-types-with-boxes}} definition is encountered. A recursive data type is a type that contains itself, such as a tree node. The solution is to use dynamic memory allocation instead of static. Rust offers a particular type of pointers for this purpose known as smart pointers, such as \boxptr{}, \rc{}, etc.

As \rrbtree{} employs structural sharing, several tree instances might point to the same sub-tree. In other words, one node can be referenced by several parent nodes simultaneously. Even though \boxptr{} enables recursive types, it does not allow shared ownership of the underlying value. A smart pointer that supports a notion of shared ownership is known as \rc{} or reference counting pointer.

\subsection{Reference counting pointers}
In a nutshell, a reference counting pointer allows shared ownership of the object wrapped into it. Every time a potential owner needs a reference to the value, the pointer itself is cloned instead of the underlying object. The reference count is incremented on each clone, and decremented when a pointer goes out of the scope. If the reference count reaches zero, the underlying value is destroyed.

\subsubsection*{Copy-on-write semantics}
Rust’s \rc{} conforms to the ownership and borrowing rules that are enforced during runtime rather than compile-time. It allows shared immutable access to the value as well as regular reference does, and permits unique mutable access only if other references do not exist.

The \mintinline{rust}{Rc::make_mut} method allows us to safely acquire a mutable pointer to the value regardless of the reference count. If there are no other pointers to the value, then \mintinline{rust}{Rc::make_mut} immediately returns a mutable reference. Otherwise, it clones the inner value to a new allocation to ensure unique ownership and then returns a reference to it. This mechanism will be referred to as conditional \emph{copy-on-write} behavior.\todo{citation for copy-on-write semantics}

\subsubsection*{Path copying}
\todo{NTH: a diagram demonstrating semantics of rc}

When updating the tree, the path leading from the root node to the affected leaf is copied to preserve the original tree from changes. The copy-on-write semantics of \rc{} is the foundation for the path copying algorithm implementation in \rrbtree{}.

Rust permits updating objects only through mutable references. To update a value in the tree, one has to acquire a mutable reference to each node that forms a path to the value. Since nodes are decorated with \rc{} pointer, a safe way to acquire a mutable reference is by calling the \mintinline{rust}{Rc::make_mut} method. If the tree has been cloned prior to the update, this call will copy each node while descending from the root to the leaf node, effectively performing \emph{path copying}.

\subsubsection*{Thread safety}
% The standard library also offers an \emph{Arc<T>} type, which is very similar to Rc<T> with the difference that it uses atomic operations to synchronize accesses to its reference counts. This can make Arc<T> a little more expensive during run time, but it enables threads to share a value safely.

% As a persistent vector is intended to be a part of a library, Arc would be a better choice as it creates a foundation for the data structure to be used in the multithreaded environment.

\subsection{Structures, enumerations and cache locality}
With the knowledge of how \rc{} helps to manage the memory and its semantics we can move on to the definition of the \rrbtree{} node using the following Rust constructs:
\begin{itemize}
    \item Structures or \emph{structs}: is a custom data type that lets you package together multiple related values\footnote{\url{https://doc.rust-lang.org/book/ch05-00-structs.html}}.
    \item Enumerations or \emph{enums}: allows you to define a type by enumerating its possible \emph{variants}\footnote{\url{https://doc.rust-lang.org/book/ch06-00-enums.html}}.
\end{itemize}

Conceptually, the \rrbtree{} consists of three node types: a balanced branch, a relaxed branch and a leaf. Depending on the tree level, branching nodes reference either other branching nodes or leaves. Thus, the node definition has to be generic over the child node type.

Rust forbids inheritance in structs. It does, however, provide alternative ways of defining data types that are conceptually related: enumerations and \emph{trait objects}\footnote{todo}.

Trait objects are objects that share common behavior by implementing the same interface. The main advantage of trait objects is their extensibility. Enums, on the other hand, cannot be extended with more variants outside of their declaration. Since the tree node is an implementation detail of a persistent vector and is not expected to be extended, enums were favored over trait objects.

Each enum variant can have a different set of fields. Hence, the enum size is capped by its largest variant to guarantee that all variants can be used interchangeably.

\subsubsection*{The \rrbtree{} node}

By using structs and enums in combination with \rc{}, we can finally bring everything together to define the \rrbtree{} node. The figure \ref{fig:rrbtree-node} presents the structure as it is in the library.

\begin{figure}[!htbp]
    \centering

    \begin{minted}{rust}
        #[cfg(feature = "arc")]
        type SharedPtr<T> = Arc<T>;

        #[cfg(not(feature = "arc"))]
        type SharedPtr<T> = Rc<T>;

        // ^ SharedPtr<T> is a type alias that is assigned
        // either Arc<T> or Rc<T> depending on the
        // thread-safety requirements and configuration.

        enum Node<T> {
            RelaxedBranch(SharedPtr<RelaxedBranch<T>>),
            Branch(SharedPtr<Branch<T>>),
            Leaf(SharedPtr<Leaf<T>>),
        }

        struct RelaxedBranch<T> {
            children: [Option<Node<T>>; BRANCH_FACTOR],
            sizes: [Option<usize>; BRANCH_FACTOR],
            len: usize,
        }

        struct Branch<T> {
            children: [Option<Node<T>>; BRANCH_FACTOR],
            len: usize,
        }

        struct Leaf<T> {
            elements: [Option<T>; BRANCH_FACTOR],
            len: usize,
        }
    \end{minted}

    \caption{Definition of the \rrbtree{} node.}
    \label{fig:rrbtree-node}
\end{figure}

Each node type has received its own struct definition. An eagle eyed reader will notice that the \type{Node} enum in figure \ref{fig:rrbtree-node} could have been declared in a more concise way by having the node fields defined directly within the variant. Even though it seems to be a more intuitive approach, it comes at a hidden cost.

\paragraph{The enum size and the cache locality}

\begin{figure}[H]
    \centering
    \begin{tikzpicture}[
        font=\ttfamily,
        nodes={
            draw,
            anchor=center,
            minimum height=8mm,
            text depth=.5ex,
            text height=2ex
        },
    ]
        \node[draw, fill=blue!20, anchor=center, minimum width=14mm, minimum height=8mm, inner sep=4pt] at (current page.north west) (rc) {ptr};
        \draw(rc.north) node[above, draw=none] {\mintinline{rust}{Rc<T>}};

        \matrix[inner sep=4pt] (rcptr) [below = 6mm of rc.south] {
            \node[fill=green!20, minimum width=14mm] {strong}; &
            \node[fill=green!20, minimum width=14mm] {weak}; &
            \node[fill=red!20, minimum width=26mm] {T}; \\
        };

        \begin{scope}[on background layer]
            \draw[fill=yellow!40] (rcptr.north west) rectangle (rcptr.south east);
        \end{scope}

        \draw[->, shorten >= 4pt] (rc.south) to (rcptr.north);

        \matrix[draw=none, inner sep=4pt] (enum-var-1) [below=16mm of rcptr.south] {
            \node[fill=green!20, minimum width=14mm] {tag}; &
            \node[fill=red!20, minimum width=24mm] {A}; &
            \node[fill=white, minimum width=16mm] {}; \\
        };
        \draw(enum-var-1.north) node[above, draw=none] {\mintinline{rust}{enum {A, B, C}}};

        \matrix[draw=none, inner sep=4pt] (enum-var-2) [below=0pt of enum-var-1.south] {
            \node[fill=green!20, minimum width=14mm] {tag}; &
            \node[fill=red!20, minimum width=40mm] {B}; \\
        };
        \matrix[draw=none, inner sep=4pt] (enum-var-3) [below=0pt of enum-var-2.south] {
            \node[fill=green!20, minimum width=14mm] {tag}; &
            \node[fill=red!20, minimum width=14mm] {C}; &
            \node[fill=white, minimum width=26mm] {}; \\
        };

        \node[draw, fill=blue!20, minimum width=14mm, minimum height=8mm, inner sep=4pt] [right=12mm of rcptr.east] (legptr) {ptr};
        \draw(legptr.east) node[right, draw=none] {4/8 bytes};

        \node[draw, fill=green!20, minimum width=14mm, minimum height=8mm, inner sep=4pt] [below=6mm of legptr.south west, anchor=west] (legsize) {size};
        \draw(legsize.east) node[right, draw=none] {4/8 bytes};

        \node[draw, fill=red!20, minimum width=20mm, minimum height=8mm, inner sep=4pt] [below=6mm of legsize.south west, anchor=west] (legt) {T};
        \draw(legt.east) node[right, draw=none] {User defined type};

        \node[draw, fill=yellow!40, minimum width=32mm, minimum height=8mm, inner sep=4pt] [below=6mm of legt.south west, anchor=west] (legalloc) {allocation};
        \draw(legalloc.east) node[right, draw=none] {Heap allocation};
    \end{tikzpicture}

    \caption{The memory layout of Rust containers.}
    \label{fig:memory-layout-of-rust-containers}
\end{figure}

As mentioned earlier, enums take up as much space as their largest variant. If the node fields were declared within the enum variants directly, the enum size would be equal to the size of \mintinline{rust}{RelaxedBranch}, resulting in the balanced tree reserving as much space as if it was relaxed. See an example of how the enum size is calculated in figure \ref{fig:memory-layout-of-rust-containers}.

The more space nodes use, the more expensive the memory allocations are. Also, bigger nodes are more likely to not fit into the CPU cache lines, negatively impacting the performance even more.

To avoid the unnecessarily large memory footprint of the \mintinline{rust}{Node} enum, the decision was made to extract the node fields into structs as demonstrated in the figure \ref{fig:rrbtree-node}. By wrapping the struct instances into a smart pointer such as \rc{}, we explicitly move the allocation of the node to the heap.

The size of the \rc{} pointer is independent of the type it encapsulates. Hence, the size of the \mintinline{rust}{Node} enum boils down to the size of the enum tag, the weak and strong reference count fields of \rc{}, and the actual pointer to the heap.

This, in turn, means that the \mintinline{rust}{Node} enum variants will be equally sized, taking as little space as possible. When compiling the library for a machine with the 64-bit CPU architecture, the enum size is 16 bytes, where 8 bytes are reserved for the enum tag, and 8 bytes for the reference-counted pointer.

\begin{figure}[!htbp]
    \centering
    \begin{tikzpicture}[
        font=\ttfamily,
        nodes={
            draw,
            anchor=center,
            minimum height=8mm,
            text depth=.5ex,
            text height=2ex
        },
    ]
        \node[draw, fill=blue!20, minimum width=14mm, minimum height=8mm, inner sep=4pt] at (current page.north east) (legptr) {ptr};
        \draw(legptr.east) node[right, draw=none] {4/8 bytes};

        \node[draw, fill=green!20, minimum width=14mm, minimum height=8mm, inner sep=4pt] [below=6mm of legptr.south west, anchor=west] (legsize) {size};
        \draw(legsize.east) node[right, draw=none] {4/8 bytes};

        \node[draw, fill=red!20, minimum width=20mm, minimum height=8mm, inner sep=4pt] [below=6mm of legsize.south west, anchor=west] (legt) {T};
        \draw(legt.east) node[right, draw=none] {User defined type};

        \node[draw, fill=yellow!40, minimum width=32mm, minimum height=8mm, inner sep=4pt] [below=6mm of legt.south west, anchor=west] (legalloc) {allocation};
        \draw(legalloc.east) node[right, draw=none] {Heap allocation};

        \matrix[draw=none, inner sep=4pt] [below left=4mm and 8mm of legalloc.west] (enum-var-relaxed-branch) {
            \node[fill=green!20, minimum width=14mm] {tag}; &
            \node[fill=blue!20, minimum width=48mm] {\mintinline{rust}{Rc<RelaxedBranch<T>>}}; \\
        };
        \draw(enum-var-relaxed-branch.north) node[above, draw=none] {\mintinline{rust}{enum Node}};

        \matrix[draw=none, inner sep=4pt] (enum-var-branch) [below=0pt of enum-var-relaxed-branch.south] {
            \node[fill=green!20, minimum width=14mm] {tag}; &
            \node[fill=blue!20, minimum width=48mm] {\mintinline{rust}{Rc<Branch<T>>}}; \\
        };
        \matrix[draw=none, inner sep=4pt] (enum-var-leaf) [below=0pt of enum-var-branch.south] {
            \node[fill=green!20, minimum width=14mm] {tag}; &
            \node[fill=blue!20, minimum width=48mm] {\mintinline{rust}{Rc<Leaf<T>>}}; \\
        };

        \matrix[draw=none, inner sep=4pt] (relaxed-branch) [below right=12mm and 2mm of enum-var-leaf.south] {
            \node[fill=green!20, minimum width=14mm] {strong}; &
            \node[fill=green!20, minimum width=14mm] {weak}; &
            \node[fill=red!20, minimum width=22mm] {children}; &
            \node[fill=red!20, minimum width=14mm] {sizes}; &
            \node[fill=red!20, minimum width=14mm] {len}; \\
        };
        \begin{scope}[on background layer]
            \draw[fill=yellow!40] (relaxed-branch.north west) rectangle (relaxed-branch.south east);
        \end{scope}

        \matrix[draw=none, inner sep=4pt] (branch) [below=8mm of relaxed-branch.south west, anchor=west] {
            \node[fill=green!20, minimum width=14mm] {strong}; &
            \node[fill=green!20, minimum width=14mm] {weak}; &
            \node[fill=red!20, minimum width=22mm] {children}; &
            \node[fill=red!20, minimum width=14mm] {len}; \\
        };
        \begin{scope}[on background layer]
            \draw[fill=yellow!40] (branch.north west) rectangle (branch.south east);
        \end{scope}

        \matrix[draw=none, inner sep=4pt] (leaf) [below=8mm of branch.south west, anchor=west] {
            \node[fill=green!20, minimum width=14mm] {strong}; &
            \node[fill=green!20, minimum width=14mm] {weak}; &
            \node[fill=red!20, minimum width=22mm] {elements}; &
            \node[fill=red!20, minimum width=14mm] {len}; \\
        };
        \begin{scope}[on background layer]
            \draw[fill=yellow!40] (leaf.north west) rectangle (leaf.south east);
        \end{scope}

        \draw[->, shorten >= 4pt, out=0, in=90] (enum-var-relaxed-branch.east) to (relaxed-branch.north);
        \draw[->, shorten >= 4pt, out=0, in=180, looseness=2] (enum-var-branch.east) to (branch.west);
        \draw[->, shorten >= 4pt, out=0, in=0, looseness=2] (enum-var-leaf.east) to (leaf.east);
    \end{tikzpicture}

    \caption{The memory layout of the \rrbtree{} node.}
    \label{fig:memory-layout-of-rrbtree-node}
\end{figure}

The figure \ref{fig:memory-layout-of-rrbtree-node} is the visualization of how the enum and struct definitions of \rrbtree{} node are arranged in memory.

\section{Pay only for the features you use}

% todo: take text from another chapter: conclusion, for example?

\subsection{Transience as the language feature}
Popular persistent data structure libraries, such s immutable-js, or collections in the standard library of Clojure, often provide an interface that is radically different from the interface of the ephemeral data structures. For example:

\begin{figure}[!htbp]

    \centering
    \begin{minted}{rust}
        let mut pvec_1 = PersistentVec::new();
        let pvec_2 = vec_1.push(42);
        //  ^ a vector instance containing 42

        let mut vec_1 = Vec::new();
        vec_1.push(42);
    \end{minted}

    \caption{A common interface of a persistent vector.}
    \label{fig:common-interface}
\end{figure}

As demonstrated, instead of updating the vector in place the operation returns a new instance that contains the change. When collection is setup like that, the library can guarantee that original vector will stay unmodified, eliminating a whole class of bugs such as race conditions.

% * This design has its own flaws, such as performance hits due to the redundant copies created when constructing a vector
% * The idea of transience was born in Clojure's programming language /ref{thesis}
% * Instead of throwing out unused copies, don't create them in the first place
% * Treating a persistent data structure as if it was mutable under safe conditions

Rust's compiler, however, eliminates this class of bugs by forbidding simultaneous mutation of a value by enforcing ownership and borrowing rules. The ownership rule essentially means that object can be owned only by a single entity at a time. Borrowing rules state that any number of immutable references can be created given that there is no \emph{mutable} references existing. Further than that, if a mutable reference exists, Rust ensures that it is the only \emph{unique} reference.

This leads us to a question if the traditional persistent collection interface is needed in Rust. The short answer is no, because compiler guards developers from common mistakes. Nicholas Matsakis, the lead engineer behind the Rust's compiler, has discussed this question in his blog\footnote{\url{http://smallcultfollowing.com/babysteps/blog/2018/02/01/in-rust-ordinary-vectors-are-values/}}.

% * Okay, Rust's compiler can defend us from those bugs, what's next?
% * So we know that too many copies is bad for performance
% * Instead of having a persistent interface, we can have a standard one, that behaves as if data structure was mutable.
% * This way we will avoid generating copies when not necessary and Rust's compiler will make sure that it is safe

% * How is this related to transience? Is this transience? How transience is different?
% * Okay, it is not, how would you call it then? Unique access optimization, because it relies on unique, mutable references for updates
% * Provide an example of how it looks like, and what are the cost of operations

% \todo{statically choosing representations: compiler flags + type aliases}
% \todo{encourages experimentation with different vector types and their performance}
% \todo{a feature that you don't need to pay for are efficient append / split operations}

\subsection{Dynamic representation}
\todo{how do vectors change their representation when concatenated or splitted}

A unique property of \pvec{} is that it features an interface identical to the standard vector, contrary to the conventional persistent interface that is significantly different.

If two vector types have identical interfaces, in theory, they can be used interchangeably without any additional changes made to the program. Hence, it becomes possible to dynamically choose a vector type depending on the context and performance requirements.

The dynamic representation explores this idea by using both the standard and tree-based vectors to ensure the best possible performance, depending on the use case.

\subsubsection*{Deciding when to switch the representation}
The standard vector is very fast in almost all operations due to its efficient, hardware friendly design. Exceptions are the operations that rely on copying, such as concat, split, and clone. The tree-based vector, on the other hand, offers effectively \bigo{1} clones and good all-around performance.

To take the best of both worlds, a vector can start as the standard vector and when it is cloned, it will be transformed into the tree-based one. Essentially, by switching the representation only when cloning, we offset the cost of using trees until they are beneficial\footnote{Nicholas Matsakis proposed a variation of this idea as future work in his blog: \url{http://smallcultfollowing.com/babysteps/blog/2018/02/01/in-rust-ordinary-vectors-are-values/}}. This optimization fits into the overall "Pay only for the features you use" idea, as it prevents the user from paying the abstraction cost until it is used.

\subsubsection*{Switching the representation}
Essentially, a persistent vector type -- \pvec{}, can be backed by two different vector types at runtime. It starts as \stdvec{}, and then switches to the \rrbvec{} on the first clone. The process of switching from one vector type to another is called \emph{spilling}, as it maps a contiguous chunk of memory into smaller pieces that are used as leaf nodes to build \rrbvec{}.

\begin{figure}[!htbp]

    \centering
    \begin{minted}{rust}
        struct PVec<T>(Representation<T>);

        enum Representation<T> {
            Flat(Vec<T>),
            Tree(RrbVec<T>),
        }
    \end{minted}

    \caption{Memory layout of \pvec{}.}
    \label{fig:pvec-memory-representation}
\end{figure}

\pvec{} has two representations: \type{Flat} that is using \stdvec{}, and \type{Tree} that is backed by \rrbvec{}. The initial capacity of both representations is equal to the branching factor of 32 by default.

\begin{figure}[!htbp]

    \centering
    \begin{tikzpicture} [
        node/.style={
            matrix of nodes,
            nodes={draw, minimum width=6mm, minimum height=8mm, anchor=center},
            font=\ttfamily,
            nodes in empty cells,
        },
        edge/.style={->, shorten >= 4pt}
    ]

        \node[] (representation-flat) at (current page.north west) { \mintinline{rust}{Representation::Flat} };
        \matrix[node, fill=green!10, inner sep=0pt] (vec) [below=6mm of representation-flat.south west, anchor=west] { 0 & 1 & 2 & 3 & 4 & 5 & 6 & 7 & 8 & 9 \\ };

        \draw[|-|]([yshift=-4mm,xshift=1mm]vec-1-1.south west) -- node (c1) [below,font=\scriptsize,outer sep=0mm] {1} ([yshift=-4mm,xshift=-1mm]vec-1-4.south east);
        \draw[|-|]([yshift=-4mm,xshift=1mm]vec-1-5.south west) -- node (c2) [below,font=\scriptsize,outer sep=0mm] {2} ([yshift=-4mm,xshift=-1mm]vec-1-8.south east);
        \draw[|-|]([yshift=-4mm,xshift=1mm]vec-1-9.south west) -- node (c3) [below,font=\scriptsize,outer sep=0mm] {3} ([yshift=-4mm,xshift=-1mm]vec-1-10.south east);

        \node[draw, dashed, inner sep=4mm, fit=(representation-flat) (c2) (vec) (vec)] (representation-flat-box) {};

        \node[] (representation-tree) [below=42mm of representation-flat.south west, anchor=west] { \mintinline{rust}{Representation::Tree} };
        \matrix[node, fill=color-node, inner sep=0pt] (node-1-1) [below=6mm of representation-tree.south] { 00 & 01 & & \\ };
        \draw(node-1-1.north) node[above, draw=none] {root};

        \matrix[node, fill=green!10, inner sep=0pt] (tail) [right=12mm of node-1-1.east] { 8 & 9 & & \\ };
        \draw(tail.north) node[above, draw=none] {tail};

        \matrix[node, fill=green!10, inner sep=0pt] (node-2-2) [below=8mm of node-1-1.south] { 4 & 5 & 6 & 7 \\ };
        \matrix[node, fill=green!10, inner sep=0pt] (node-2-1) [left=2mm of node-2-2.west] { 0 & 1 & 2 & 3 \\ };

        \draw[edge, out=225, in=45] (node-1-1-1-1.south) to (node-2-1.north);
        \draw[edge, out=225, in=45] (node-1-1-1-2.south) to (node-2-2.north);

        \draw[|-|]([yshift=-4mm,xshift=1mm]node-2-1-1-1.south west) -- node (b1) [below,font=\scriptsize,outer sep=0mm] {1} ([yshift=-4mm,xshift=-1mm]node-2-1-1-4.south east);
        \draw[|-|]([yshift=-4mm,xshift=1mm]node-2-2-1-1.south west) -- node (b2) [below,font=\scriptsize,outer sep=0mm] {2} ([yshift=-4mm,xshift=-1mm]node-2-2-1-4.south east);
        \draw[|-|]([yshift=-4mm,xshift=1mm]tail-1-1.south west) -- node (b3) [below,font=\scriptsize,outer sep=0mm] {3} ([yshift=-4mm,xshift=-1mm]tail-1-4.south east);

        \node[draw, dashed, inner sep=4mm, fit=(representation-tree) (b2) (tail) (node-2-1)] (representation-tree-box) {};

        \draw[edge, out=270, in=90] (representation-flat-box.south) to (representation-tree-box.north);
    \end{tikzpicture}

    \caption{Visualization of how the vector representation is switched.}
    \label{fig:switching-representations}
\end{figure}

Spilling happens by converting the flat vector into an iterator of sub-arrays that are 32 elements wide and pushing them directly onto \rrbtree{} as leaf nodes. If the last chunk is smaller than 32, then it will be used as the tail for the new \rrbvec{}. See figure \ref{fig:switching-representations} that visualizes the transition.

One could have considered splitting up the original vector into chunks and reusing them instead of copying. Unfortunately, heap allocators do not have a mechanism for safely splitting an existing allocation into smaller ones. The Rust's allocator interface\footnote{\url{https://doc.rust-lang.org/std/alloc/struct.System.html}} reflects this limitation by exposing functions that allow allocating and freeing memory but no subdivision of existing allocations.

Additionally, the cost of spilling needs to be paid only once, and will be amortized by \bigo{1} consequent clones.

% TODO: Spatial locality, and its subclass sequential locality. Both terms can be used to frame the argument around performance of RrbVec iterators. You **have** to make this point in the discussion chapter on this.
% TODO: a modification of the display and focus optimization was used to improve performance of \rrbvec{}'s iterator:
% The \rrbtree{} iterator does not implement display optimization as described in \ref{todo}. It does, however, consume the tree by chunks. Avoiding the tree traversal on every request to return the next element offsets its cost, and results in overall amortized(effectively?)-constant runtime.